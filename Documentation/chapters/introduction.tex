\chapter{Introduction}

\section[Historical backgrounds]{Historical backgrounds\footnote{This introduction is rewritten after the school project has passed. I've declared this project to be independent of the school ever since. So, it's going to be much more authentic and straightforward.}}

Light, the first gateway to the quantum world of the humankind. I've always been intrigued by it. And, we've been able to confirm that light can be entangled, causing correlations in its polarization axes. Most modern development of physics has been into that field. But often, we tend to forget what we left behind four hundred years ago: the principle of least action. What happens when light passes through different medium? Refraction, everyone can answer that. But what happens if light is entangled?

To resolve this problem, I originally turned to the Schr\"odinger's equation for two particles. However, it's quite complex, and the mathematics behind it doesn't really elude me that much. So, I turned to the path integrals method. Its ability to intuitively produce the classical world, governed by the principle of least action, is captivating. The action principle appears so naturally in path integrals that I thought I was hallucinating. Question arises: what does the principle of least action looks like for a many body systems? Is it any different if they're entangled? I was literally mashing random words together at that point, but it has led me into the realm of physics that I don't think anyone really cares before: the dynamics of an entangled system.

Entanglement is usually defined as the inseparability of states, commonly used to describe the spin of a particle, which has been studied in great extent. It has found its use in quantum computer, giving its blazing speed. However, from my knowledge, no literatures has mentioned the effect of entanglement on the movement of particles. Will entangled particle move together in the same direction? If I separate a pair of entangled particles, put one in a potential, and let the other be free, does the particle in the potential have any effect on the free particle? And thus, this research (\emph{project}) was born.

This research started out by two people; me, and the other one that shall not be named. However, I have a lot of conflicts with that person. Long story short, he quitted; and thus, I'm on my own journey.

\printunsrtglossary[type = symbols, style = longright, title = {List of Symbols (Still incomplete)}]