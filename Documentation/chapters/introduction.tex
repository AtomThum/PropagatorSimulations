\chapter{Introduction}

\paragraph{Comment on the introduction} This introduction is rewritten after the school project has passed. I've declared this project to be independent of the school ever since. So, it's going to be much more authentic and straightforward.

\section{Academical background}

\section{Historical background}

I'm captured by the path integrals method of quantum mechanics, and its ability to intuitively produce the classical world, which is governed by the principle of least action. It appears so naturally in path integrals that I thought I was hallucinating. I wasn't. However, one question arises in my mind: what does the principle of least action looks like for more than one particle? And what if they're entangled? I was literally mashing random words together at that point, but it has led me into the realm of physics that I don't think anyone really cares before: the dynamics of an entangled system.

Entanglement is usually defined as the inseparability of states, commonly used to describe the spin. Spin entanglement has been studied in great extent, and has been abused in quantum computer to achieve its blazing fast speed. From my knowledge however, no literatures has mentioned what entanglement would do to the movement of the particle. Will entangled particle move together in the same direction? If I separate a pair of entangled particles, put one in a potential, and let the other be free, does the particle in the potential have any effect on the free particle? And thus, this research (\emph{project}) was born.

This research was originally planned for two people; the other one shall not be named. However, I have a lot of conflicts with him, so he quitted; and thus, I'm on my own journey.