\chapter{Propagator and joint probability distribution}

\section{General theory for the propagator of two particles}

\subsection{Non-interacting systems}

The propagator for a single particle system is used to find the probability distribution $\tilde{P}(x)$ of a state at anytime, i.e., for a state $\ket{\psi(t_0)}$,
\begin{equation}
    \tilde{P}(q_F, t_F) = \braket{q_F}{\psi(t_F)} = \int(q_0, \Dd{t})\psi(q_0, t_0)\dd{q_0}K.
\end{equation}
When taken the modulus squared, $|\tilde{P}(q_F, t_F)|^2$, or just $P(q_F, t_F)$, represents the probability of finding that particle at the point $(q_F, t_F)$ in spacetime.

We shall then extend this idea to describe a two particle system. Let there be a state ket $\ket{\eta(t)}$, which simultaneously represents the state of both particle one and particle two. Assume that at some point in time, $\ket{\eta}$ is separable:
\begin{equation}
    \ket{\eta(t_0)} = \ket{\psi(t_0)} \tensor \ket{\phi(t_0)}.
\end{equation}
If the subsystem of $\ket{\eta}$ is non-interacting, the Hamiltonian must be completely separable, which also implies that the time evolution operator $\hat{U}$ is also separable. Consider $\braket{q_F, q_F'}{\eta(t_F)}$ where the unprimed $q$ belongs to the position in the Hilbert space of particle one, and the primed $q$'s, in Hilbert space of particle two.
\begin{align*}
    \braket{q_F, q_F'}{\eta(t_F)} &= \bra{q_F, q_F'}\left(\ket{\psi_i(t_F)}\ket{\phi_i(t_F)}\right) \\
    &= \psi_i(q_F, t_F)\phi_i(q_F', t_F') \\
    &= \iint K_{\psi}(q_F, t_F; q_0, t_0)K_{\phi}(q_F', t_F'; q_0', t_0')\left[\psi_i(q_F, t_F)\phi_i(q_F', t_F')\right] \dd{q_0}\dd{q_0'}.
\end{align*}
For most systems, $K_{\psi}$ and $K_{\phi}$ is identical because both $\psi_i$ and $\phi_i$ is affected by the same potential. But for some, $K_{\psi}$ and $K_{\phi}$ might not be identical. E.g., if $\psi$ is in a potential well, but $\phi$ is a free particle that's infinitely far away. $K_{\psi}$ must be of the potential well, and $K_{\phi}$ must be of the free particle.

It can be seen that $\braket{q_F, q_F'}{\eta(t_F)}$ represents the probability distribution of both the two particles. E.g., for a state ket $\ket{\psi}\tensor\ket{\phi}$,
\begin{align}
    & (\bra*{q_F}\bra*{q_F'})(\ket*{\psi(t_F)}\ket*{\phi(t_F')}) \\
    &= \braket*{q_F}{\psi(t_F)}\braket*{q_F'}{\phi(t_F')} \label{eq:prejointprobability1}\\
    &= \iint\dd{q_0}\dd{q_0'}K(q_F, t_F; q_0, t_0)K(q_F', t_F';q_0', t_0')(\psi(q_0, t_0)\phi(q_0', t_0')) \\
    &= \int\dd{q_0}K(q_F, t_F; q_0, t_0)\psi(q_0, t_0)\int\dd{q_0'}K(q_F', t_F';q_0', t_0')\phi(q_0', t_0'), \label{eq:prejointprobability2}
\end{align}
\cref{eq:prejointprobability1} says that $\braket{q_F, q_F'}{\eta(t_F)}$ is the product of the probability that $\psi$ is at $(q_F, t_F)$, and $(q_F', t_F')$. Therefore, $\braket{q_F, q_F'}{\eta(t_F)}$ represents the joint probability distribution of the two states, and it can be found via the product of propagators,
\begin{equation}
    K(q_F, t_F; q_0, t_0)K(q_F', t_F';q_0', t_0').
\end{equation}


\subsection{Interacting systems}

Let there be a state ket $\ket{\eta}$ which describes the quantum state of two particles. When the two particles are interacting, the time-evolution operator cannot be factored into tensor products of two operators. Therefore, there can only be one combined time-evolution operator for both of the systems:
\begin{equation}
    \hat{U}(t_F, t_0) = \exp[-\im\Dd{t}\left(\frac{\hat{p}^2}{2m} + \frac{\hat{p}'^2}{2m'} + V(\hat{q}, \hat{q}', t)\right)]
\end{equation}
The consequence of the combined time-evolution operator is that, we cannot calculate the joint probability distribution between different time of the subsystem. For simplicity’ sake, let the index $j = i + 1$, where $t_j = t_i + \Dd{t}$ in which $\Dd{t} \appr 0$.
\begin{align}
    &\braket{q_j, q_j'}{\eta(t_j)} \\
    &= \mel{q_j, q_j'}{\hat{U}(t_j, t_i)}{\eta(t_i)}. \\
    &= \iint \dd{q_i}\dd{q_i'} \mel{q_j, q_j'}{\hat{U}(t_j, t_i)}{q_i, q_i'}\braket{q_i, q_i'}{\eta(t_i)}.
\end{align}

As seen, the form of transition element is
\begin{equation}
    \mel{q_j, q_j'}{\exp[-\im\Dd{t}\left(\frac{\hat{p}^2}{2m} + \frac{\hat{p}'^2}{2m'} + V(\hat{q}, \hat{q}', t)\right)]}{q_i, q_i'}.
\end{equation}
The terms in the exponents can be separated due to the vanishing commutator when $\Dd{t} \appr 0$. By separating the terms in the exponents and inserting two complete sets of momentum basis, the equation above turns into
\begin{align}
    & \frac{1}{(2\cpi)^2} \iint\dd{p}\dd{p'} \mel{q_j, q_j'}{\exp[-\im\Dd{t}\frac{\hat{p}^2}{2m}]}{p, p'}\mel{p, p'}{\exp[-\im\Dd{t}\frac{\hat{p}'^2}{2m'}]\exp[-\im\Dd{t}V(\hat{q}, \hat{q}', t)]}{q_i, q_i'} \nonumber\\
    &= \frac{1}{(2\cpi)^2} \iint\dd{p}\dd{p'} \exp[-iu\Dd{t}\left(\frac{p^2}{2m} + \frac{p'^2}{2m'} + V(q_i, q_i, \Dd{t})\right)]\braket{q_j}{p}\braket{q_j'}{p'}\braket{q_i}{p}\conj\braket{q_j'}{p'}\conj \nonumber\\
    &= \frac{1}{(2\cpi)^4} \e^{-\im\Dd{t}V(q_i, q_i', t)} \int\dd{p}\exp[-\im\Dd{t}\frac{p^2}{2m} + \im p(q_j - q_i)]\int\dd{p'}\exp[-\im\Dd{t}\frac{p'^2}{2m'} + \im p'(q_j' - q_i')^2] \nonumber\\
    &= \frac{(mm')^{\frac{1}{2}}}{8\cpi^3\im\Dd{t}}\exp[-\im\Dd{t}V(q_i, q_i', t) + \frac{\im m'}{2\Dd{t}}(q_j' - q_i')^2 + \frac{\im m}{2\Dd{t}}(q_j - q_i)^2].
\end{align}
To find the propagator for an interacting system, we need to perform successive integrals on $q$ and $q'$, i.e.,
\begin{equation}
    K_{\eta} = \idotsint \dd{q_N}\dd{q_N'}\dots\dd{q_1}\dd{q_1'}\mel{q_F, q_F'}{\hat{U}(t_F, t_N)}{q_N, q_N'}\dots\mel{q_1, q_1'}{\hat{U}(t_1, t_0)}{q_0, q_0'}
\end{equation}

Notice that when there is no interaction between the two systems ($V = 0$), the integrals become separable and reduces down to the form of the non-interacting system's propagator but off by a normalization factor.

There are two common forms of interaction, which is the spring interaction and the coulomb interaction. Both of which includes the term $(q_i - q_i')^2$ in $V(q, q')$, which causes major problems in integration. When expanded, there is a $q_iq_i'$ term that makes the integral inseparable which causes the integral pattern to not repeat; therefore, we resort to perturbation.

\section{Form of the two particle perturbation series}

The perturbation series for one particle are already given by Feynman in his path integrals textbook:
\begin{equation}
    K_n(F, 0) = \left(-\im\right)^n\idotsint K_0(F, n)\prod_{i = 1}^n V(i)K(i, i - 1)\dd{\tau_i}
\end{equation}
where $K(j, k) = K(q_j, t_j; q_k, t_k)$. To apply it with two particles, we extend it:
\begin{equation}
    K_n(F, 0; F', 0') = \left(-\im\right)^n\idotsint K_0(F, n)K_0(F', n')\left[\prod_{i = 1}^nV(i)K(i, i - 1)K(i', i - 1')\dd{q_i}\dd{q_i'}\right]\dd{t}. \label{eq:generaltwoparticleperturbation}
\end{equation}
where $K = \ssum K_n$. For some potential, it is possible to evaluate this series term by term analytically using the Meijer's $G$ function.

\section{The spring system}

The spring potential is given by $V(q, q') = \flatfrac{k}{2}(q - q')^2$, which I shall let $\alpha = \flatfrac{k}{2}$ for simplicity sake.

\subsection{First order perturbation term}
\label{sec:spring_1storder}

From \cref{eq:generaltwoparticleperturbation}, set $t_1 = t, t_0 = 0$
\begin{align}
    &K_1(F, 0; F', 0) \nonumber \\
    &= \left(-\im\right)^1\iiint K_0(F, 1)K_0(F', 1')K_0(1, 0)K_0(1, 0')\alpha(q_1 - q_1')^2\dd{q_1}\dd{q_1'}\dd{t}. \\
    &= -\im\alpha\int_0^{t_F}\left[\iint K_0(q_F, q_1; t_F - t)K_0(q_F', q_1'; t_F - t)K_0(q_1, q_0; t)K_0(q_1', q_0'; t)(q_1 - q_1')^2\dd{q_1}\dd{q_1'}\right]\dd{t}. \nonumber
\end{align}
We then let the terms in the square bracket,
\begin{equation}
    \iint K_0(q_F, q_1; t_F - t)K_0(q_F', q_1'; t_F - t)K_0(q_1, q_0; t)K_0(q_1', q_0'; t)(q_1 - q_1')^2\dd{q_1}\dd{q_1'}
\end{equation}
equals $I$; therefore, $K_1 = -\im\alpha\int_0^{t_F} I\dd{t}$.

We then separate $I$ into three integrals:
\begin{gather}
    I_{P1} = \int q_1^2K_0(q_F, q_1; t_F - t)K_0(q_1, q_0; t)\dd{q_1} \int K_0(q_F', q_1'; t_F, t)K_0(q_1', q_0'; t)\dd{q_1'}, \label{eq:spring_IP1}\\
    I_{P2} = \int K_0(q_F, q_1; t_F - t)K_0(q_1, q_0; t) \int q_1'^2K_0(q_F', q_1'; t_F, t)K_0(q_1', q_0'; t)\dd{q_1'}, \label{eq:spring_IP2}\\
    I_{P3} = \int q_1K_0(q_F, q_1; t_F - t)K_0(q_1, q_0; t) \int q_1'K_0(q_F', q_1'; t_F, t)K_0(q_1', q_0'; t) \dd{q_1'} \label{eq:spring_IP3}
\end{gather}
where $I = I_{P1} + I_{P2} + 2I_{P3}$. The integrals without the factor $q_1$ and $q_1^2$ can be reduced into the kernel for the free particle:
\begin{gather}
    I_{P1} = K_0(q_F', q_0'; t_F)\int q_1^2K_0(q_F, q_1; t_F - t)K_0(q_1, q_0; t)\dd{q_1} \\
    I_{P2} = K_0(q_F, q_0; t_F)\int q_1'^2 K_0(q_F', q_1'; t_F - t)K_0(q_1', q_0'; t)\dd{q_1'}
\end{gather}
Since $I_{P2}$ can be obtained by switching all the primed variables with the corresponding unprimed in $I_{P1}$, we're left with two family of integrals:
\begin{equation}
    \int q_1K_0(q_F, q_1; t_F - t)K_0(q_1, q_0; t)\dd{q_1} \mathand \int q_1K_0(q_F, q_1; t_F - t)K_0(q_1, q_0; t)\dd{q_1}. \label{eq:spring_family_of_integrals}
\end{equation}
To evaluate these, we first simplify the product of kernel under the assumption that $t_F > t$.
\begin{align}
    &K_0(q_F, q_1; t_F - t)K_0(q_1, q_0; t) \nonumber\\
    &= \sqrt{\frac{m}{2\cpi\im(t_F - t)}}\sqrt{\frac{m}{2\cpi\im t}}\exp[-\frac{\im m}{2(t_F - t)}(q_F - q_1)^2 - \frac{\im m}{2t}(q_1 - q_0)^2] \\
    &= \frac{m}{2\cpi}\sqrt{-\frac{1}{t(t_F - t)}} \exp[q_1^2\left(\frac{\im m}{2t} + \frac{\im m}{2(t_F - t)}\right) - q_1\left(\frac{\im m q_0}{t} + \frac{\im m q_F}{t_F - t}\right) + \left(\frac{\im m q_0^2}{2t} + \frac{\im m q_F^2}{2(t_F - t)}\right)] \nonumber \\
    &= \frac{m}{2\cpi}\sqrt{-\frac{1}{t(t_F - t)}} \exp[-q_1^2\left(\frac{m t_F}{2\im t(t_F - t)}\right) - q_1(\im m)\left(\frac{q_0}{t} + \frac{q_F}{t_F - t}\right) - \left(\frac{m}{2\im}\right)\left(\frac{q_0^2}{t} + \frac{q_F^2}{t_F - t}\right)] \nonumber
\end{align}
The normalization factor are pulled out. Both integrals in \cref{eq:spring_family_of_integrals} can be evaluated with
\begin{equation}
    a = \frac{m t_F}{2\im t(t_F - t)}, \quad b = \im m\left(\frac{q_0}{t} + \frac{q_F}{t_F - t}\right) \mathand c = \left(\frac{m}{2\im}\right)\left(\frac{q_0^2}{t} + \frac{q_F^2}{t_F - t}\right)
\end{equation}
in which,
\begin{equation}
    \exp[\frac{b^2}{4a} - c] = \exp[\frac{\im m}{2t_F}(q_F - q_0)^2] = \sqrt{\frac{2\cpi\im t_F}{m}}K_0(q_F, q_0; t_F).
\end{equation}
To summarize,
\begin{align}
    \int q_1K_0(q_F, q_1; t_F - t)K_0(q_1, q_0; t)\dd{q_1} &= -\frac{b}{2}\sqrt{\frac{\cpi}{a^3}}\times\frac{m}{2\cpi}\sqrt{-\frac{1}{t(t_F - t)}}\times\sqrt{\frac{2\cpi\im t_F}{m}}K_0(q_F, q_0; t_F) \nonumber\\
    &= -\frac{b}{2}\sqrt{\frac{\cpi}{a^3}} \times \sqrt{\frac{mt_F}{2\cpi\im t(t_F - t)}}K_0(q_F, q_0; t_F), \label{eq:spring_deg1gaussian_form}
\end{align}
and
\begin{equation}
    \int q_1^2K_0(q_F, q_1; t_F - t)K_0(q_1, q_0; t)\dd{q_1} = \frac{1}{4}\sqrt{\frac{\cpi}{a^5}}(2a + b^2) \times \sqrt{\frac{mt_F}{2\cpi\im t(t_F - t)}}K_0(q_F, q_0; t_F). \label{eq:spring_deg2gaussian_form}
\end{equation}

On the Gaussian integral with degree one,
\begin{equation}
    -\frac{b}{2}\sqrt{\frac{\cpi}{a^3}} = \sqrt{\frac{2\cpi t}{mt_F^3}}\frac{\sqrt{-\im(t_F - t)^3}}{\im(t_F - t)} \times \left[q_0(t_F - t) + q_Ft\right];
\end{equation}
thus from \cref{eq:spring_deg1gaussian_form},
\begin{align}
    \int q_1K_0(q_F, q_1; t_F - t)K_0(q_1, q_0; t) &= \begin{multlined}[t]
        \sqrt{\frac{2\cpi t}{mt_F^3}}\frac{\sqrt{-\im(t_F - t)^3}}{\im(t_F - t)} \times \left[q_0(t_F - t) + q_Ft\right] \\
        \times \sqrt{\frac{mt_F}{2\cpi\im t(t_F - t)}}K_0(q_F, q_0; t_F)
    \end{multlined} \\
    &= -\frac{1}{t_F}\left[q_0(t_F - t) + q_Ft\right]K_0(q_F, q_0; t_F).
\end{align}
On the Gaussian integral with degree two,
\begin{equation}
    \frac{1}{4}\sqrt{\frac{\cpi}{a^5}}(2a + b^2) = -\sqrt{\frac{2\cpi t}{m^3t_F^5}}\frac{\sqrt{\im(t_F - t)^5}}{(t_F - t)^2}\left[m\left(q_0(t_F - t) + q_Ft\right)^2 + \im tt_F(t_F - t)\right];
\end{equation}
and thus,
\begin{align}
    &\int q_1^2K_0(q_F, q_1; t_F - t)K_0(q_1, q_0; t)\dd{q_1} \\
    &= \sqrt{\frac{mt_F}{2\cpi\im t(t_F - t)}}K_0(q_F, q_0; t_F)  \times -\sqrt{\frac{2\cpi t}{m^3t_F^5}}\frac{\sqrt{\im(t_F - t)^5}}{(t_F - t)^2}\left[m\left(q_0(t_F - t) + q_Ft\right)^2 + \im tt_F(t_F - t)\right] \nonumber\\
    &= -\frac{1}{mt_F^2}\left[m\left(q_0(t_F - t) + q_Ft\right)^2 + \im tt_F(t_F - t)\right]K_0(q_F, q_0; t_F).
\end{align}

We're now in the place to finally construct the first order propagator term for the spring system. Recall that
\begin{equation}
    K_1 = -\im\alpha\int_0^{t_F} I\dd{t} = -\im\alpha\left[\int_0^{t_F}I_{P1}\dd{t} + \int_0^{t_F}I_{P2}\dd{t} + 2\int_0^{t_F}I_{P3}\dd{t}\right]
\end{equation}
From earlier,
\begin{gather}
    \begin{aligned}
        I_{P1} &= K_0(q_F', q_0'; t_F)\int q_1^2K_0(q_F, q_1; t_F - t)K_0(q_1, q_0; t)\dd{q_1} \\
        &= -\frac{1}{mt_F^2}\left[m\left(q_0(t_F - t) + q_Ft\right)^2 + \im tt_F(t_F - t)\right]K_0(q_F, q_0; t_F)K_0(q_F', q_0'; t_F),
    \end{aligned}
    \\
    \begin{aligned}
        I_{P2} &= K_0(q_F, q_0; t_F)\int q_1'^2 K_0(q_F', q_1'; t_F - t)K_0(q_1', q_0'; t)\dd{q_1'} \\
        &= -\frac{1}{mt_F^2}\left[m\left(q_0'(t_F - t) + q_F't\right)^2 + \im tt_F(t_F - t)\right]K_0(q_F, q_0; t_F)K_0(q_F', q_0'; t_F),
    \end{aligned}
    \\
    \begin{aligned}
        I_{P3} &= -\frac{1}{t_F}\left[q_0(t_F - t) + q_Ft\right]K_0(q_F, q_0; t_F) \times -\frac{1}{t_F}\left[q_0'(t_F - t) + q_F't\right]K_0(q_F', q_0'; t_F) \\
        &= \frac{1}{t_F^2}\left[q_0(t_F - t) + q_Ft\right]\left[q_0'(t_F - t) + q_F't\right]K_0(q_F, q_0; t_F)K_0(q_F', q_0'; t_F).
    \end{aligned}
\end{gather}
Thus,
\begin{multline}
    K_1 = -\im\alpha K_0(q_F, q_0; t_F)K_0(q_F', q_0'; t_F) \frac{1}{t_F^2} \left[-\frac{1}{m}\int_0^{t_F}m\left(q_0(t_F - t) + q_Ft\right)^2 + \im tt_F(t_F - t)\dd{t} \right. \\
    -\frac{1}{m}\int_0^{t_F}m\left(q_0'(t_F - t) + q_F't\right)^2 + \im tt_F(t_F - t)\dd{t} \\
    \left. + \int_0^{t_F}\left[q_0(t_F - t) + q_Ft\right]\left[q_0'(t_F - t) + q_F't\right]\dd{t}\right]
\end{multline}
The integrals are then taken out to be evaluated term by term. Since the integrand of these integrals are all polynomials, we can just plug it into \texttt{SymPy.jl}:
\begin{gather}
    \int_0^{t_F}m\left(q_0(t_F - t) + q_Ft\right)^2 = \frac{t_F^3}{6}\left(2m(q_0 + q_F)^2 + \im t_F\right) \\
    \int_0^{t_F}m\left(q_0'(t_F - t) + q_F't\right)^2 = \frac{t_F^3}{6}\left(2m(q_0' + q_F')^2 + \im t_F\right) \\
    \int_0^{t_F}\left[q_0(t_F - t) + q_Ft\right]\left[q_0'(t_F - t) + q_F't\right]\dd{t} = \frac{t_F^3}{6}\left(2q_0q_0' + q_0q_F' + q_0'q_F + 2q_Fq_F'\right)
\end{gather}
Therefore, the first order perturbation term of the spring system takes the form
\begin{multline}
    K_1(q_F, q_0; q_F', q_0'; t_F) = -\im\frac{\alpha t_F}{6}K_0(q_F, q_0; t_F)K_0(q_F', q_0'; t_F) \\
    \times\left[-2\left(m(q_0 + q_F)^2 + m(q_0' + q_F')^2 + \im t_F\right) + 2q_0q_0' + q_0q_F' + q_0'q_F + 2q_Fq_F'\right].
\end{multline}

\subsection{Second order perturbation term}

The second order perturbation term, $K_2$ is
\begin{align}
    K_2 &= \idotsint K_0(F, 2)V(2)K_0(2, 1)V(1)K_0(1, 0)\dd{q_1}\dd{q_1'}\dd{t_1}\dd{q_2}\dd{q_2'}\dd{t_2} \\
    &= \begin{multlined}[t]
        \idotsint K_0(q_F, q_2; t_F - t_2)K_0(q_F', q_2'; t_F - t_2)K_0(q_2, q_1; t_2 - t_1)K_0(q_2, q_1; t_2 - t_1) \\ \times K_0(q_1, q_0; t_1 - t_0)K_0(q_1, q_0; t_1 - t_0)(q_2 - q_2')^2(q_1 - q_1')^2\dd{q_1}\dd{q_1'}\dd{t_1}\dd{q_2}\dd{q_2'}\dd{t_2}
    \end{multlined} \\
    &= \begin{multlined}[t]
        \int_{t_1}^{t_F}\int_{t_0}^{t_F}\left[\idotsint K_0(q_F, q_2; t_F - t_2)K_0(q_F', q_2'; t_F - t_2)K_0(q_2, q_1; t_2 - t_1)\right. \\ \times K_0(q_2, q_1; t_2 - t_1)K_0(q_1, q_0; t_1 - t_0)K_0(q_1, q_0; t_1 - t_0) \left(\vphantom{\left(q_{2}^{\prime}\right)^{2}} q_{1}^{2} q_{2}^{2} - 2 q_{1}^{2} q_{2} q_{2}^{\prime} \right.\\ + q_{1}^{2} \left(q_{2}^{\prime}\right)^{2} - 2 q_{1} q_{2}^{2} q_{1}^{\prime} + 4 q_{1} q_{2} q_{1}^{\prime} q_{2}^{\prime} - 2 q_{1} q_{1}^{\prime} \left(q_{2}^{\prime}\right)^{2} + q_{2}^{2} \left(q_{1}^{\prime}\right)^{2} \\ \left.\left. - 2 q_{2} \left(q_{1}^{\prime}\right)^{2} q_{2}^{\prime} + \left(q_{1}^{\prime}\right)^{2} \left(q_{2}^{\prime}\right)^{2} \right) \dd{q_1}\dd{q_1'}\dd{q_2}\dd{q_2'} \vphantom{\idotsint}\right]\dd{t_1}\dd{t_2}
    \end{multlined}
\end{align}
The integral once again can be broken into nine integrals that must be integrated w.r.t. time twice later on. All those nine integrals have a product of propagator as a multiplier. We shall evaluate those first, separating the primed and unprimed variables.
\begin{align}
    &\begin{multlined}[t]
        K_0(q_F, q_2; t_F - t_2)K_0(q_F', q_2'; t_F - t_2)K_0(q_2, q_1; t_2 - t_1)K_0(q_2, q_1; t_2 - t_1)K_0(q_1, q_0; t_1 - t_0) \\ \times K_0(q_1, q_0; t_1 - t_0)
    \end{multlined} \nonumber \\
    &= \begin{multlined}[t]
        \frac{\im m^3}{8\cpi^3(t_F - t_2)(t_2 - t_1)(t_1 - t_0)}\exp \left[\frac{\im m}{2(t_F - t_2)}\left((q_F - q_2)^2 + (q_F' - q_2')^2\right) \right. \\ \left. + \frac{\im m}{2(t_2 - t_1)}\left((q_2 - q_1)^2 + (q_2' - q_1')^2\right) + \frac{\im m}{2(t_1 - t_0)}\left((q_1 - q_0)^2 + (q_1' - q_0')^2\right)\right]
    \end{multlined}
\end{align}

\subsection{The joint probability distribution}

\section{The delta function collision problem}

The potential for the delta function collision problem is
\begin{equation}
    V = V_0\ddel(q - q')
\end{equation}
where $V_0$ is the strength of the delta function, and is generally considered to be negative.

\subsection{First order perturbation term}

We evaluate the perturbation term similarly to how we did it in \cref{sec:spring_1storder}, starting with the form:
\begin{align}
    &K_1(F, 0; F', 0) \\
    &= -\im\int_0^{t_F}\left[\iint K_0(q_F, q_1; t_F - t)K_0(q_F', q_1'; t_F - t)K_0(q_1, q_0; t)K_0(q_1', q_0'; t)V_0\ddel(q_1 - q_1')\dd{q_1}\dd{q_1'}\right]\dd{t}. \nonumber \\
    &= -\im V_0\int_0^{t_F}\left[\iint K_0(q_F, q_1; t_F - t)K_0(q_F', q_1'; t_F - t)K_0(q_1, q_0; t)K_0(q_1', q_0'; t)\ddel(q_1 - q_1')\dd{q_1}\dd{q_1'}\right]\dd{t} \nonumber
\end{align}
Let $I_1$ represents the integral in the square bracket; thus, $K_1 = -\im V_0\int_{0}^{t_F}I_1\dd{t}$. Then,
\begin{align}
    I_1 &= \iint K_0(q_F, q_1; t_F - t)K_0(q_F', q_1'; t_F - t)K_0(q_1, q_0; t)K_0(q_1', q_0'; t)\ddel(q_1 - q_1')\dd{q_1}\dd{q_1'} \\
    &= \int K_0(q_F, q_1'; t_F - t)K_0(q_F', q_1'; t_F - t)K_0(q_1', q_0; t)K_0(q_1', q_0'; t)\dd{q_1'}
\end{align}

\section{The coulomb problem}