\chapter{The generalized Gaussian integrals}

A Gaussian integral is the integral of the Gaussian function $\exp[-x^2]$:
\begin{equation}
    \int_{\infty}^{-\infty} \exp[-x^2]\dd{x} = \sqrt{\cpi}.
\end{equation}
However, the path integral method usually demands a more generalized form of the Gaussian integral denoted in this study with $\Omega_n$:
\begin{equation}
    \Omega_n(a, b, c; x) \equiv \int_{\infty}^{-\infty} x^n\exp[-(ax^2 + bx + c)]\dd{x},\label{eq:generalized_gaussian}
\end{equation}
where $a$, $b$, and $c$ are constants and $n$ is a positive integer. Because we're mostly going to be dealing with indefinite integrals, all integrals without a bound is assumed to be evaluated from $-\infty$ to $\infty$.

To evaluate $\Omega_n$ (\cref{eq:generalized_gaussian}), we shall complete the square first, then substitute the exponents to fit the form of the standard Gaussian integral.
\begin{align*}
    \int x^n\e^{-(ax^2 + bx + c)}\dd{x} &= \int x^n\exp[-a\left(x + \frac{b^2}{2a}\right)^2 + \frac{b^2}{4a} - c]\dd{x} \\
    &= \e^{\frac{b^2}{4a} - c}\int x^n\exp[-a\left(x + \frac{b^2}{2a}\right)]\dd{x}.
\end{align*}
Let
\begin{align}
    -u^2 &= -a\left(x + \frac{b^2}{2a}\right)^2 \\
    x &= \frac{u}{\sqrt{a}} - \frac{b^2}{2a} \\
    \dv{x}{u} &= \dv{u}(\frac{u}{\sqrt{a}} - \frac{b^2}{2a}) \\
    \dd{x} &= \frac{1}{\sqrt{a}}\dd{u}.
\end{align}
Then,
\begin{align}
    \Omega_n &= \e^{\frac{b^2}{4a} - c}\frac{1}{\sqrt{a}}\int\left(\frac{u}{\sqrt{a}} - \frac{b^2}{2a}\right)^n\e^{-u^2}\dd{u}. \\
    &= \e^{\frac{b^2}{4a} - c}\frac{1}{\sqrt{a}}\int\sum_{k = 0}^{n}{n \choose k}\left(\frac{b^2}{2a}\right)^{n - k}\left(\frac{u}{\sqrt{a}}\right)^k\e^{-u^2}\dd{u}. \\
    &= \e^{\frac{b^2}{4a} - c}\frac{1}{\sqrt{a}}\sum_{k = 0}^{n}\left[{n \choose k}\left(\frac{b^2}{2a}\right)^{n - k}\left(\frac{1}{\sqrt{a}}\right)^{k} \int u^k\e^{u^2}\dd{u}\right], \label{eq:generalized_gaussian-1}
\end{align}
i.e., $\Omega_n$ can be written as a sum of
\begin{equation}
    \omega = \int u^k\e^{-u^2}\dd{u}.
\end{equation}
Notice that when $n$ is odd, the integrand of $\omega$ is an odd function; thus, $\omega$ is zero when $n$ is odd. When $n$ is even however, the integrand is even and thus can be simplified into
\begin{equation}
    \omega = 2\int_{0}^{\infty} u^{2m}\e^{-u^2}\dd{u}
\end{equation}
where $k = 2m$. We then do another substitution by letting $-t = -u^2$. Thus, $u = \sqrt{t}$ and $\dd{u} = \flatfrac{\dd{t}}{2\sqrt{t}}$. The bound of the integral isn't affected by a radical. Our integral then becomes
\begin{align}
    \omega &= 2 \times \int_{0}^{\infty}t^{m}\e^{-t}\frac{1}{2\sqrt{t}}\dd{t}. \\
    &= \int_{0}^{\infty}t^{m - \frac{1}{2}}\e^{-t}\dd{t} \\
    &= \int_{0}^{\infty}t^{m + \frac{1}{2} - 1}\e^{-t}\dd{t} \\
    &= \fgamma\left(m + \frac{1}{2}\right) \\
    &= \fgamma\left(\frac{k + 1}{2}\right).
\end{align}
Since the integral evaluates to the gamma function for only even numbers, we add a term that makes $\omega = 0$ when $n$ is odd:
\begin{equation}
    \omega = \frac{1}{2}((-1)^k + 1)\fgamma\left(\frac{k + 1}{2}\right).
\end{equation}
Thus, from \cref{eq:generalized_gaussian-1},
\begin{equation}
    \Omega_n = \e^{\frac{b^2}{4a} - c}\frac{1}{\sqrt{a}}\sum_{k = 0}^{n}\left[{n \choose k}\left(\frac{b^2}{2a}\right)^{n - k}\left(\frac{1}{\sqrt{a}}\right)^{k} \frac{1}{2}((-1)^k + 1)\fgamma\left(\frac{k + 1}{2}\right)\right]
\end{equation}
To continue this, we then expand the combinatorics and use the formulas of arguments with half-integer real part to get
\begin{align}
    &\begin{multlined}
    \e^{\frac{b^2}{4a} - c}\frac{1}{\sqrt{a}}\sum_{k = 0}^{n}\frac{n!}{k!(n - k)!}\left(\frac{b^2}{2a}\right)^n\left(\frac{2a}{b^2}\right)^k\left(\frac{1}{\sqrt{a}}\right)^k\left(\sqrt{\cpi}\frac{k!}{4^{\frac{k}{2}}\left(\flatfrac{k}{2}\right)!}\right) \\
    \times \frac{1}{2}((-1)^k + 1)
    \end{multlined} \\
    &= \e^{\frac{b^2}{4a} - c}n!\sqrt{\frac{\cpi}{a}}\left(\frac{b^2}{2a}\right)^n\sum_{k = 0}^n\frac{1}{(n - k)!(\flatfrac{k}{2})!}\left(\frac{\sqrt{a}}{b^2}\right)^{k}\frac{1}{2}((-1)^k + 1)
\end{align}
To simplify the summation, we replace $k$ with $2k$, and the upper limit with $m$ where $m = \left\lfloor\flatfrac{n}{2}\right\rfloor$. Thus, we get
\begin{equation}
    \left[\sum_{k = 0}^{m}\frac{1}{(n - 2k)!k!}\left(\frac{\sqrt{a}}{b^2}\right)^{2k}\right]\e^{\frac{b^2}{4a} - c}n!\left(\frac{b^2}{2a}\right)^n\sqrt{\frac{\cpi}{a}}
\end{equation}
As a convention in this study, I shall define
\begin{equation}
    \int_{\infty}^{\infty}x^n\exp[-ax^2 - bx - c]\dd{x} = \Omega_n(a, b, c; x) \equiv \zeta_n(a, b, c)\kappa(a, b, c) 
\end{equation}
where
\begin{gather}
    \zeta_n(a, b, c) \equiv \left[\sum_{k = 0}^{m}\frac{1}{(n - 2k)!k!}\left(\frac{\sqrt{a}}{b^2}\right)^{2k}\right]n!\left(\frac{b^2}{2a}\right)^n\sqrt{\frac{\cpi}{a}}, \\
    \kappa(a, b, c) \equiv \exp[\frac{b^2}{4a} - c].
\end{gather}
I shall refer to $\zeta$ as the coefficient of the Gaussian integral, and $\kappa$, the resulting exponent. The following section is a table for values of $\zeta_n$ from $n = 1$ to $10$. 

\paragraph{Comment on the generalized Gaussian integral} The generalized Gaussian integral can be cast in the form of Meijer's $G$ function. For $n \geq 0$,
\begin{multline}
    \int x^n\exp[-ax^2 - bx - c]\dd{x} \\= \frac{e^{-c}}{2\sqrt{\cpi}ab}\left[- \frac{2^{n + 1}a}{b^n} {G_{2, 1}^{1, 2}\left(\begin{matrix} \frac{1 - n}{2}, - \frac{n}{2}  \\0 \end{matrix} \middle| {\frac{4 a e^{- 2 i \pi}}{b^{2}}} \right)} + a^{\frac{1 - n}{2}}b {G_{1, 2}^{2, 1}\left(\begin{matrix} \frac{1 - n}{2}\\0, \frac{1}{2}\end{matrix} \middle| {\frac{b^{2}}{4 a}} \right)}\right]
\end{multline}
However, this form isn't that helpful because the Meijer's $G$ function is defined with integral of products.

Originally, I was going to evaluate everything using \texttt{SymPy}. However, \texttt{SymPy} doesn't bother integrating any Gaussian integrals that have complex arguments. My own implementation is written in \cref{sec:implementation_of_simulation} 

\section{Tables of Gaussian integrals with varying orders}
\label{sec:table_of_gaussian_integrals}

\everymath{\displaystyle}
\begin{enumerate}[label = {$\zeta_\arabic*$:}]
    \setcounter{enumi}{-1}
    \item $+\frac{1}{1}\sqrt{\frac{\cpi}{a}}$
    \item $-\frac{b}{2}\sqrt{\frac{\cpi}{a^3}}$
    \item $+\frac{1}{4}\sqrt{\frac{\cpi}{a^5}} (2a + b^2)$
    \item $-\frac{b}{8}\sqrt{\frac{\cpi}{a^7}} (6a + b^2)$
    \item $+\frac{1}{16}\sqrt{\frac{\cpi}{a^9}} (12a^2 + 12ab^2 + b^4)$
    \item $-\frac{b}{32}\sqrt{\frac{\cpi}{a^{11}}} (60a^2 + 20ab^2 + b^2)$
    \item $+\frac{1}{64}\sqrt{\frac{\cpi}{a^{13}}} (120a^3 + 180a^2b^2 + 30ab^4 + b^6)$
    \item $-\frac{b}{128}\sqrt{\frac{\cpi}{a^{15}}} (840a^3 + 420a^2b^2 + 42ab^4 + b^6)$
    \item $+\frac{1}{256}\sqrt{\frac{\cpi}{a^{17}}} (1680a^4 + 3360a^3b^2 + 840a^2b^4 + 56ab^6 + b^8)$
    \item $-\frac{b}{512}\sqrt{\frac{\cpi}{a^{19}}} (15120a^4 + 10080a^3b^2 + 1512a^2b^4 + 72ab^6 + b^8)$
\end{enumerate}