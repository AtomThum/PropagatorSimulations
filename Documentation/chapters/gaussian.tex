\chapter{The generalized Gaussian integrals}

\section{Preliminary form}

A Gaussian integral is the integral of the Gaussian function $\exp[-x^2]$:
\begin{equation}
    \int_{\infty}^{-\infty} \exp[-x^2]\dd{x} = \sqrt{\cpi}.
\end{equation}
However, this is not that useful for path integrals, because it demands a more generalized form of this Gaussian function. Therefore, we focus on integrals of the form
\begin{equation}
    I_1 = \int_{\infty}^{-\infty} x^n\exp[-(ax^2 + bx + c)]\dd{x},\label{eq:generalized_gaussian}
\end{equation}
where $a$, $b$, and $c$ are constants and $n$ is a positive integer. This integral will be very useful later on when evaluating the Born expansion of the propagator.

Because we're just going to be dealing with integrals from negative to positive infinity, I shall omit the bounds of the integrals. Thus, every integral written with no bounds from now on are assumed to be a definite integral from negative to positive infinity.

\section{Evaluation}

To evaluate $I_1$ (\cref{eq:generalized_gaussian}), we shall complete the square first, then substitute the exponents to fit the form of the standard Gaussian integral.
\begin{align*}
    \int x^n\e^{-(ax^2 + bx + c)}\dd{x} &= \int x^n\exp[-a\left(x + \frac{b^2}{2a}\right)^2 + \frac{b^2}{4a} - c]\dd{x} \\
    &= \e^{\frac{b^2}{4a} - c}\int x^n\exp[-a\left(x + \frac{b^2}{2a}\right)]\dd{x}.
\end{align*}
Let
\begin{align}
    -u^2 &= -a\left(x + \frac{b^2}{2a}\right)^2 \\
    x &= \frac{u}{\sqrt{a}} - \frac{b^2}{2a} \\
    \dv{x}{u} &= \dv{u}(\frac{u}{\sqrt{a}} - \frac{b^2}{2a}) \\
    \dd{x} &= \frac{1}{\sqrt{a}}\dd{u}.
\end{align}
Then,
\begin{align}
    I_1 &= \e^{\frac{b^2}{4a} - c}\frac{1}{\sqrt{a}}\int\left(\frac{u}{\sqrt{a}} - \frac{b^2}{2a}\right)^n\e^{-u^2}\dd{u}. \\
    &= \e^{\frac{b^2}{4a} - c}\frac{1}{\sqrt{a}}\int\sum_{k = 0}^{n}{n \choose k}\left(\frac{b^2}{2a}\right)^{n - k}\left(\frac{u}{\sqrt{a}}\right)^k\e^{-u^2}\dd{u}. \\
    &= \e^{\frac{b^2}{4a} - c}\frac{1}{\sqrt{a}}\sum_{k = 0}^{n}\left[{n \choose k}\left(\frac{b^2}{2a}\right)^{n - k}\left(\frac{1}{\sqrt{a}}\right)^{k} \int u^k\e^{u^2}\dd{u}\right], \label{eq:generalized_gaussian-1}
\end{align}
i.e., $I_1$ can be written as a sum of
\begin{equation}
    I_0 = \int u^k\e^{-u^2}\dd{u}.
\end{equation}
Notice that when $n$ is odd, the integrand of $I_0$ is an odd function. Therefore, $I_0$ is zero whenever $n$ is odd. When $n$ is even however, the integrand is even. Thus, it can be simplified to
\begin{equation}
    2\int_{0}^{\infty} u^{2m}\e^{-u^2}\dd{u}
\end{equation}
where $k = 2m$. We then do another substitution by letting $-t = -u^2$. Thus, $u = \sqrt{t}$ and $\dd{u} = \flatfrac{\dd{t}}{2\sqrt{t}}$. Both infinity and zero aren't affected by a square root, therefore the bound doesn't change. Our integral then becomes
\begin{align}
    &2 \times \int_{0}^{\infty}t^{m}\e^{-t}\frac{1}{2\sqrt{t}}\dd{t}. \\
    &= \int_{0}^{\infty}t^{m - \frac{1}{2}}\e^{-t}\dd{t} \\
    &= \int_{0}^{\infty}t^{m + \frac{1}{2} - 1}\e^{-t}\dd{t} \\
    &= \fgamma\left(m + \frac{1}{2}\right) \\
    &= \fgamma\left(\frac{k + 1}{2}\right).
\end{align}
Since the integral evaluates to the gamma function for only even numbers, we add a term that makes $I_0 = 0$ when $n$ is odd:
\begin{equation}
    I_0 = \frac{1}{2}((-1)^k + 1)\fgamma\left(\frac{k + 1}{2}\right).
\end{equation}
Thus, from \cref{eq:generalized_gaussian-1},
\begin{equation}
    I_1 = \e^{\frac{b^2}{4a} - c}\frac{1}{\sqrt{a}}\sum_{k = 0}^{n}\left[{n \choose k}\left(\frac{b^2}{2a}\right)^{n - k}\left(\frac{1}{\sqrt{a}}\right)^{k} \frac{1}{2}((-1)^k + 1)\fgamma\left(\frac{k + 1}{2}\right)\right]
\end{equation}
To continue this, we then expand the combinatorics and use the formulas of arguments with half-integer real part to get
\begin{align}
    &\begin{multlined}
    \e^{\frac{b^2}{4a} - c}\frac{1}{\sqrt{a}}\sum_{k = 0}^{n}\frac{n!}{k!(n - k)!}\left(\frac{b^2}{2a}\right)^n\left(\frac{2a}{b^2}\right)^k\left(\frac{1}{\sqrt{a}}\right)^k\left(\sqrt{\cpi}\frac{k!}{4^{\frac{k}{2}}\left(\flatfrac{k}{2}\right)!}\right) \\
    \times \frac{1}{2}((-1)^k + 1)
    \end{multlined} \\
    &= \e^{\frac{b^2}{4a} - c}n!\sqrt{\frac{\cpi}{a}}\left(\frac{b^2}{2a}\right)^n\sum_{k = 0}^n\frac{1}{(n - k)!(\flatfrac{k}{2})!}\left(\frac{\sqrt{a}}{b^2}\right)^{k}\frac{1}{2}((-1)^k + 1)
\end{align}
To simplify the summation, we replace $k$ with $2k$, and the upper limit with $m$ where $m = \left\lfloor\flatfrac{n}{2}\right\rfloor$. Thus, we get
\begin{equation}
    \left[\sum_{k = 0}^{m}\frac{1}{(n - 2k)!k!}\left(\frac{\sqrt{a}}{b^2}\right)^{2k}\right]\e^{\frac{b^2}{4a} - c}n!\left(\frac{b^2}{2a}\right)^n\sqrt{\frac{\cpi}{a}}
\end{equation}
As far as I know, this equation cannot be simplified further.

\paragraph{Comment on the generalized Gaussian integral} Note that the integral in \cref{eq:generalized_gaussian} can be casted in the form of Meijer's $G$ function for $n \geq 0$:
\begin{multline}
    \int x^n\exp[-ax^2 - bx - c]\dd{x} \\= \frac{e^{-c}}{2\sqrt{\cpi}ab}\left[- \frac{2^{n + 1}a}{b^n} {G_{2, 1}^{1, 2}\left(\begin{matrix} \frac{1 - n}{2}, - \frac{n}{2}  \\0 \end{matrix} \middle| {\frac{4 a e^{- 2 i \pi}}{b^{2}}} \right)} + a^{\frac{1 - n}{2}}b {G_{1, 2}^{2, 1}\left(\begin{matrix} \frac{1 - n}{2}\\0, \frac{1}{2}\end{matrix} \middle| {\frac{b^{2}}{4 a}} \right)}\right]
\end{multline}
However, this form doesn't seem to be very helpful when we try to evaluate the polynomial part of the Gaussian integral because of the nature of the Meijer's $G$ function that's defined with integral of products.

Originally, I was going to evaluate everything using \texttt{SymPy}, and not really care about anything else. However, I came to realize that \texttt{SymPy} doesn't bother integrating any Gaussian integrals that have complex number components. My own implementation will be written in \cref{sec:implementation_of_simulation} 

\section{Tables of Gaussian integrals with varying orders}
\label{sec:table_of_gaussian_integrals}

\everymath{\displaystyle}
\begin{enumerate}[label = {$n = \arabic*$:}]
    \setcounter{enumi}{-1}
    \item $+\frac{1}{1}\sqrt{\frac{\cpi}{a}}\e^{\frac{b^2}{4a} - c}$
    \item $-\frac{b}{2}\sqrt{\frac{\cpi}{a^3}} \e^{\frac{b^2}{4a} - c}$
    \item $+\frac{1}{4}\sqrt{\frac{\cpi}{a^5}} (2a + b^2) \e^{\frac{b^2}{4a} - c}$
    \item $-\frac{b}{8}\sqrt{\frac{\cpi}{a^7}} (6a + b^2) \e^{\frac{b^2}{4a} - c}$
    \item $+\frac{1}{16}\sqrt{\frac{\cpi}{a^9}} (12a^2 + 12ab^2 + b^4)\e^{\frac{b^2}{4a} - c}$
    \item $-\frac{b}{32}\sqrt{\frac{\cpi}{a^{11}}} (60a^2 + 20ab^2 + b^2)\e^{\frac{b^2}{4a} - c}$
    \item $+\frac{1}{64}\sqrt{\frac{\cpi}{a^{13}}} (120a^3 + 180a^2b^2 + 30ab^4 + b^6)\e^{\frac{b^2}{4a} - c}$
    \item $-\frac{b}{128}\sqrt{\frac{\cpi}{a^{15}}} (840a^3 + 420a^2b^2 + 42ab^4 + b^6)\e^{\frac{b^2}{4a} - c}$
    \item $+\frac{1}{256}\sqrt{\frac{\cpi}{a^{17}}} (1680a^4 + 3360a^3b^2 + 840a^2b^4 + 56ab^6 + b^8)\e^{\frac{b^2}{4a} - c}$
    \item $-\frac{b}{512}\sqrt{\frac{\cpi}{a^{19}}} (15120a^4 + 10080a^3b^2 + 1512a^2b^4 + 72ab^6 + b^8)\e^{\frac{b^2}{4a} - c}$
\end{enumerate}