\chapter{Implementation of simulations}
\label{sec:implementation_of_simulation}

\section{The brute force approach}

To represent the joint probability amplitude of a bipartite system, we need three spatial axes. Two for the position basis of each Hilbert space, and one for the probability amplitude. From \cref{sec:general_theory_propagator_position}, the joint probability amplitude of a bipartite system represents a surface, and the volume under that surface must be normalized to one. The joint probability function $P(q_F, q_F'; t_F)$ is represented by the modulus squared of the state, projected onto the position basis.
\begin{align}
    P(q_F, q_F'; t_F) &= \left|\braket{q_F, q_F'}{\eta(t_F)}\right|^2 \\
    &= \left|\iint K(q_F, q_F'; q_0, q_0'; t_F - t_0)\eta(q_0, q_0'; t_0)\dd{q_0}\dd{q_0'}\right|^2 \\
    &= \left|\iint \left\{\sum_{i = 0}^{\infty}K_i(q_F, q_F'; q_0, q_0'; t_F - t_0)\right\}\eta(q_0, q_0', t_0)\dd{q_0}\dd{q_0'}\right|^2 \\
    \intertext{To implement this equation, we need to discretize it by swapping the integral for the summation sign.}
    &= \left|\sum_{\vphantom{q_0'}q_0}\sum_{q_0'}\left\{\sum_{\vphantom{q_0'}i = 0}^{\infty}K_i(q_F, q_F'; q_0, q_0'; t_F - t_0)\right\}\eta(q_0, q_0'; t_0)\right|^2 \label{eq:computation_brute_force}
\end{align}
The sum of propagators can then be truncated to our liking. And, here is the code in \texttt{Julia}.
\begin{minted}{julia}
using Plots; plotlyjs()
using LinearAlgebra

freePropagator(finPos, startPos, finTime, startTime = 0, m = 1) = sqrt(m / (2 * pi * im * (finTime - startTime))) * exp(im * m / (2 * (finTime - startTime)) * (finPos - startPos)^2)  
freePropagatorC(qf, qfp, q0, q0p, tf, t0) = freePropagator(qf, q0, tf, t0) * freePropagator(qfp, q0p, tf, t0)
initStateFunction(q0, q0p, σ1, s1, p1, σ2, s2, p2) = (1//2 * pi * σ1)^(1//4) * exp(-(q0 - s1)^2 / (4 * σ1^2) + im * p1 * q0) * (1//2 * pi * σ2)^(1//4) * exp(-(q0p - s2)^2 / (4 * σ2^2) + im * p2 * q0p)
springPropagator1(qf, qfp, q0, q0p, tf, t0) = freePropagatorC(qf, qfp, q0, q0p, tf, t0) * (1 - im * α * tf / 6 * (-2 * (m * (q0 + qf)^2 + m * (q0p + qfp)^2 + im * tf) + 2*q0*q0p + q0*qfp + q0p*qf + 2*qf*qfp))

# Initialization of the position basis
maxPos = 10
minPos = -10
stepPos = 0.25

pos1Vect = collect(minPos:stepPos:maxPos)
pos2Vect = collect(minPos:stepPos:maxPos)
posVectSize = size(pos1Vect, 1)
posMat = [(i, j) for i in pos1Vect, j in pos2Vect]

posToIndex(pos) = Int32((pos - minPos) / stepPos  + 1)

# Select unentangled and entangled state by commenting/uncommenting
# Unentangled
initState(q) = initStateFunction(q[1], q[2], 0.5, +1, 0, 0.5, -1, 0)
# Entangled
# initState(q) = 1/sqrt(2) * (initStateFunction(q[1], q[2], 0.5, +1, 0, 0.5, -1, 0) - initStateFunction(q[1], q[2], 0.5, -1, 0, 0.5, +1, 0))
global α
α = 1
m = 1

springPropagator1(qf, qfp, q0, q0p, tf, t0) = freePropagatorC(qf, qfp, q0, q0p, tf, t0) * (1 - im * α * tf / 6 * (-2 * (m * (q0 + qf)^2 + m * (q0p + qfp)^2 + im * tf) + 2*q0*q0p + q0*qfp + q0p*qf + 2*qf*qfp))
initMat = initState.(posMat)
initMat = round.(initMat, digits = 7)

finalMat = Matrix{ComplexF32}(undef, posVectSize, posVectSize)
finalTime = 1

for xf in pos1Vect, xfp in pos2Vect
    sumPos = 0
    for i in 1:posVectSize
        Threads.@threads for j in 1:posVectSize
            x0 = pos1Vect[i]
            x0p = pos2Vect[j]
            sumPos += freePropagatorC(xf, xfp, x0, x0p, finalTime, 0) * initMat[i, j]
        end
    end
    finalMat[posToIndex(xf), posToIndex(xfp)] = sumPos
end
surface(pos1Vect, pos2Vect, abs2.(finalMat))
\end{minted}

\section{Parallelization approach}

The method in the section above is quite computationally expensive due to the amount of for loops in there, with the time complexity of $O(n^4)$. To reduce this computation time, we must parallelize it on the \texttt{GPU}. First, let's contain $\eta(q_0, q_0'; t_0)$ in a matrix
\begin{equation}
    \hat{\eta} = \mqty[
        \eta(x_1, x_1') & \cdots & \eta(x_1, x_n') \\
        \vdots & \ddots & \vdots \\
        \eta(x_n, x_1') & \cdots & \eta(x_n, x_n') 
    ].
\end{equation}
Notice that in \cref{eq:computation_brute_force}, $(q_F, q_F')$ and $t_F$ is independent of the sum. Therefore, we can create a matrix that's filled with propagators, taking the position index of the matrix, and $(q_F, q_F'; t_F)$ as the universal input:
\begin{equation}
    \hat{K}(q_F, q_F'; t_F) = \mqty[
        K(q_F, q_F'; x_1, x_1'; t_F - t_0) & \cdots & K(q_F, q_F'; x_1, x_n'; t_F - t_0) \\
        \vdots & \ddots & \vdots \\
        K(q_F, q_F'; x_n, x_1'; t_F - t_0) & \cdots & K(q_F, q_F'; x_n, x_n'; t_F - t_0)
    ].
\end{equation}
The Hadamard product (element-wise product) of $\hat{K}(q_F, q_F'; t_F)$ and $\hat{\eta}$, denoted $\hat{K}(q_F, q_F'; t_F) \circ \hat{\eta}$, is
\begin{equation}
    \mqty[
        K(q_F, q_F'; x_1, x_1'; t_F - t_0)\eta(x_1, x_1') & \cdots & K(q_F, q_F'; x_1, x_n'; t_F - t_0)\eta(x_1, x_n') \\
        \vdots & \ddots & \vdots \\
        K(q_F, q_F'; x_n, x_1'; t_F - t_0)\eta(x_n, x_1') & \cdots & K(q_F, q_F'; x_n, x_n'; t_F - t_0)\eta(x_n, x_n')
    ].
\end{equation}
We can then take broadcast the absolute squared function into the matrix, then find the sum of all elements, resulting in
\begin{equation}
    \left|\sum_{i = 1}^{n}\sum_{j = 1}^n K(q_F, q_F'; x_i, x_j'; t_F - t_0)\eta(x_i, x_j')\right|^2,
\end{equation}
which directly equates to the sum in \cref{eq:computation_brute_force}. Using this method, all the computation can be parallelized on the \texttt{GPU}: vectorized broadcasting for the generation of $\hat{K}$, multiplication broadcasting for the Hadamard product, and array reduction using addition for the sum of all elements. We then implement it as follows.
\begin{minted}{julia}
using Plots; plotlyjs()
using LinearAlgebra
using CUDA
CUDA.allowscalar(false)

# Initialization of the position basis
maxPos = 5
minPos = -5
stepPos = 0.25

# Gaussian state function
initStateFunction(q0, q0p, σ1, s1, p1, σ2, s2, p2) = (1//2 * pi * σ1)^(1//4) * exp(-(q0 - s1)^2 / (4 * σ1^2) + im * p1 * q0) * (1//2 * pi * σ2)^(1//4) * exp(-(q0p - s2)^2 / (4 * σ2^2) + im * p2 * q0p)
posToIndex(pos) = Int32((pos - minPos) / stepPos + 1)

pos1Vect = collect(minPos:stepPos:maxPos)
pos2Vect = collect(minPos:stepPos:maxPos)
posVectSize = size(pos1Vect, 1)
posMat = [(i, j) for i in pos1Vect, j in pos2Vect]
xMesh = pos1Vect * ones(posVectSize)'
yMesh = xMesh'
xMeshCu = CuArray(xMesh)
yMeshCu = CuArray(yMesh)

# The initial state can be uncommented for an entangled state
initState(q) = initStateFunction(q[1], q[2], 0.5, +1, 0, 0.5, -1, 0)# + initStateFunction(q[1], q[2], 0.5, -3, 0, 0.5, +3, 0)
initMat = initState.(posMat)
initMat = round.(initMat, digits = 6)
initMatCu = CuArray(initMat)

global α, m, finalTime
α = 1
m = 1
finalTime = 0.01

springProp(qf, qfp, q0, q0p, tf, t0) = @. sqrt(m / (2 * pi * im * (tf - t0))) * exp(im * m / (2 * (tf - t0)) * (qf - q0)^2) * sqrt(m / (2 * pi * im * (tf - t0))) * exp(im * m / (2 * (tf - t0)) * (qfp - q0p)^2) * (1 - im * α * tf / 6 * (-2 * (m * (q0 + qf)^2 + m * (q0p + qfp)^2 + im * tf) + 2*q0*q0p + q0*qfp + q0p*qf + 2*qf*qfp))
freeProp(qf, qfp, q0, q0p, tf, t0) = @. sqrt(m / (2 * pi * im * (tf - t0))) * exp(im * m / (2 * (tf - t0)) * (qf - q0)^2) * sqrt(m / (2 * pi * im * (tf - t0))) * exp(im * m / (2 * (tf - t0)) * (qfp - q0p)^2)

finalMat = Matrix{ComplexF32}(undef, posVectSize, posVectSize)
for i in pos1Vect, j in pos2Vect
    # Generation of the propagator matrix
    propMat = springProp(i, j, xMeshCu, yMeshCu, finalTime, 0)
    # Componentwise multiplication
    propMat .*= initMatCu
    # Array reduction using addition
    finalMat[posToIndex(i), posToIndex(j)] = reduce(+, propMat)
end
# Plotting the final surface
surface(pos1Vect, pos2Vect, normalize(abs2.(finalMat)))
\end{minted}

\section{Numerical computation of the entanglement entropy and covariance}