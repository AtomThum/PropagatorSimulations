\chapter{Implementation of simulations}
\label{sec:implementation_of_simulation}

\section{The bruteforce approach}

To represent the joint probability distribution of a two particle system, we need four axes. Two for the position of each Hilbert space that those particles belong in, one for the probability amplitude, and one for the time.
\begin{align}
    P(q_F, q_F'; t_F) &= \left|\braket{q_F, q_F'}{\eta(t_F)}\right|^2 \\
    &= \left|\iint K(q_F, q_F', t_F; q_0, q_0', t_0)\eta(q_0, q_0', t_0)\dd{q_0}\dd{q_0'}\right|^2 \\
    &= \left|\iint \left\{\sum_{i = 0}^{\infty}K_i(q_F, q_F', t_F; q_0, q_0', t_0)\right\}\eta(q_0, q_0', t_0)\dd{q_0}\dd{q_0'}\right|^2 \\
    \intertext{For this equation to work with a discrete simulation, we need to exchange the integral sign with the summation sign:}
    &= \left|\sum_{\vphantom{q_0'}q_0}\sum_{q_0'}\left\{\sum_{\vphantom{q_0'}i = 0}^{\infty}K_i(q_F, q_F', t_F; q_0, q_0', t_0)\right\}\eta(q_0, q_0', t_0)\right|
\end{align}
The sum of propagators can then be truncated to our liking. I've written the code in \texttt{Julia} to test for the free particle propagator, and it looks something like this:
\begin{minted}{julia}
using Plots; plotlyjs()
using LinearAlgebra
using SymPy
using BenchmarkTools

q, q0, q1, q2, qf = symbols("q q_0 q_1 q_2 q_F", real = true)
m, t, t0, t1, t2, tf = symbols("m t t_0 t_1 t_2 t_F", real = true, positive = true)
q0p, q1p, q2p, qfp = symbols("q^{\\prime}_0 q^{\\prime}_1 q^{\\prime}_2 q^{\\prime}_F", real = true)
a, b, c, d, e, f = symbols("a b c d e f")
s1, s2, p1, p2, σ1, σ2 = symbols("s_1 s_2 p_1 p_2 σ_1 σ_2")

freePropagator(finPos, startPos, finTime, startTime = 0, m = 1) = sqrt(m / (2 * pi * im * (finTime - startTime))) * exp(im * m / (2 * (finTime - startTime)) * (finPos - startPos)^2)  
freePropagatorC(qf, qfp, q0, q0p, tf, t0) = freePropagator(qf, q0, tf, t0) + freePropagator(qfp, q0p, tf, t0)

initStateFunction(q0, q0p, σ1, s1, p1, σ2, s2, p2) = (1//2 * pi * σ1)^(1//4) * exp(-(q0 - s1)^2 / (4 * σ1^2) + im * p1 * q0) * (1//2 * pi * σ2)^(1//4) * exp(-(q0p - s2)^2 / (4 * σ2^2) + im * p2 * q0p)

maxPos = 5
minPos = -5
stepPos = 0.25

pos1Vect = collect(minPos:stepPos:maxPos)
pos2Vect = collect(minPos:stepPos:maxPos)
posVectSize = size(pos1Vect, 1)
posMat = [(i, j) for i in pos1Vect, j in pos2Vect]

posToIndex(pos) = Int32((pos - minPos) / stepPos  + 1)

initState(q) = initStateFunction(q[1], q[2], 1, 2, 2, 1, -2, -2)
initMat = initState.(posMat)
initMat = round.(initMat, digits = 7)

surface(pos1Vect, pos2Vect, abs.(initMat))

finalTime = 1
finalMat = Matrix{ComplexF32}(undef, posVectSize, posVectSize)

for xf in pos1Vect, xfp in pos2Vect
    sumPos = 0
    for i in 1:posVectSize, j in 1:posVectSize
        x0 = pos1Vect[i]
        x0p = pos2Vect[j]
        sumPos += freePropagatorC(xf, xfp, x0, x0p, finalTime, 0) * initMat[i, j]
    end
    finalMat[posToIndex(xf), posToIndex(xfp)] = sumPos
end

surface(pos1Vect, pos2Vect, abs2.(finalMat))
\end{minted}

\section{Parallelization approach}