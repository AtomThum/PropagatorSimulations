\chapter{Implementation of simulations}
\label{sec:implementation_of_simulation}

\section{The brute force approach}

To represent the joint probability amplitude of a bipartite system, we need three spatial axes. Two for the position basis of each Hilbert space, and one for the probability amplitude. From \cref{sec:general_theory_propagator_position}, the joint probability amplitude of a bipartite system represents a surface, and the volume under that surface must be normalized to one. The joint probability function $P(q_F, q_F'; t_F)$ is represented by the modulus squared of the state, projected onto the position basis.
\begin{align}
    P(q_F, q_F'; t_F) &= \left|\braket{q_F, q_F'}{\eta(t_F)}\right|^2 \\
    &= \left|\iint K(q_F, q_F', t_F; q_0, q_0', t_0)\eta(q_0, q_0', t_0)\dd{q_0}\dd{q_0'}\right|^2 \\
    &= \left|\iint \left\{\sum_{i = 0}^{\infty}K_i(q_F, q_F', t_F; q_0, q_0', t_0)\right\}\eta(q_0, q_0', t_0)\dd{q_0}\dd{q_0'}\right|^2 \\
    \intertext{To implement this equation, we need to discretize it by swapping the integral for the summation sign.}
    &= \left|\sum_{\vphantom{q_0'}q_0}\sum_{q_0'}\left\{\sum_{\vphantom{q_0'}i = 0}^{\infty}K_i(q_F, q_F', t_F; q_0, q_0', t_0)\right\}\eta(q_0, q_0', t_0)\right|
\end{align}
The sum of propagators can then be truncated to our liking. I've written the code in \texttt{Julia} to test for the free particle propagator, and it looks something like this:
\begin{minted}{julia}
using Plots; plotlyjs()
using LinearAlgebra
\end{minted}

\section{Parallelization approach}