\chapter{Integrals evaluation code}
\label{appendix:sympy_code_base}

All of these codes that I've written are in the \texttt{Julia} language, which I've imported three packages: \texttt{Sympy}, \texttt{OffsetArrays}, and \texttt{Plots; plotlyjs()}

\section{Polynomial extraction}

Since we're going to be doing a lot of polynomials rearranging, I've implemented the polynomial extraction function as follows:
\begin{minted}{julia}
function extractPolynomial(expr, arg)
    expr isa Sym ? nothing : expr = sympify(1)
    expr = expand(expr)
    polyDegree = Int(degree(expr, arg))
    extractOneSet = []
    for i in 0:polyDegree
        temp = expr.coeff(arg, i)
        push!(extractOneSet, temp)
    end
    polyExt = OffsetVector(extractOneSet, 0:polyDegree)
    return polyExt
end
\end{minted}
This function accepts two arguments: \texttt{expr}, which is the expression you want to extract, and \texttt{arg}, the variable that you extract with respect to. For example, inputting $ax^2 + bx + c,~x$ would give out
\begin{equation}
    \begin{bmatrix}
        a \\ b \\ c
    \end{bmatrix},
\end{equation}
which is a zero index matrix.

\section{Code for the spring problem}

\subsection{First order perturbation}

Solving