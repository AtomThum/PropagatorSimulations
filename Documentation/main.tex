\documentclass[a4paper, 11pt, openleft]{memoir}

\usepackage{indentfirst}

% Fonts
\usepackage[no-math]{fontspec}
\setmainfont{ZillaSlab}[
    Path = fonts/,
    Extension = .ttf,
    UprightFont = *-Regular,
    ItalicFont = *-Italic,
    BoldFont = *-Bold,
    BoldItalicFont = *-BoldItalic,
    Ligatures = {TeX, Common}
]
\defaultfontfeatures{Mapping = tex-text, Scale = MatchLowercase}
\usepackage{mathtools}
\usepackage[math-style = TeX, mathrm = sym, warnings-off = {mathtools-colon}]{unicode-math}
\setmathfont[Scale = MatchLowercase]{Concrete-Math.otf}
\setoperatorfont\symscr
\newcolumntype{C}{>{$}c<{$}}
\newcolumntype{L}{>{$}l<{$}}
\newcolumntype{R}{>{$}r<{$}}

% Page Layout and Margins
\setsecnumdepth{subsection}
\settocdepth{subsection}
\setlrmarginsandblock{4cm}{1cm}{*}
\setulmarginsandblock{3cm}{2cm}{*}
\setlength{\headheight}{24.90704pt}
\setlength{\parindent}{1.5cm}
\setlength{\parskip}{0.3em}
\setlength{\beforechapskip}{20pt}
\noDisplayskipStretch
\allowdisplaybreaks
\setSpacing{1.4}
\XeTeXlinebreakskip=0pt plus 3pt
\checkandfixthelayout

% Footnotes
\usepackage{fancyhdr}
\pagestyle{fancy}
\fancyhead[LE]{\textbf{\thepage ~ $\big\vert$ \leftmark ~(Revision: \today, Puripat Thumbanthu)}}
\fancyhead[RO]{\textbf{(Revision: \today, Puripat Thumbanthu) ~\rightmark ~ $\big\vert$ \thepage}}
\fancyhead[RE, LO, C]{}

% Indices
\usepackage{imakeidx}
\makeindex

% Packages
\usepackage[dvipsnames]{xcolor}
\usepackage{caption, subcaption, graphicx, pdfpages, float, wrapfig}
\usepackage[inkscapeversion = 1, inkscapelatex = true]{svg}
\svgpath{{diagrams/}}
\graphicspath{{diagrams/}}
\usepackage[inline]{enumitem}
\usepackage{csquotes}

\usepackage{multicol}
\setlength{\columnseprule}{1pt}
\def\columnseprulecolor{\color{lightgray}}

% Styling Figures
\DeclareCaptionLabelSeparator{pipe}{ $\vert$ }
\captionsetup{
    labelfont = {bf},
    font = {small, sc},
    width = 0.6\textwidth,
    labelsep = pipe,
    figurename = \textbf{Fig. }
}

% Styling Titles
\renewcommand{\partnamefont}{\LARGE\bfseries\scshape\centering}
\renewcommand{\partnumfont}{\LARGE\bfseries\scshape\centering\MakeUppercase}
\renewcommand{\midpartskip}{\par\rule{1in}{0.5pt}\vspace{1em}\par}
\renewcommand{\printparttitle}{\HUGE\bfseries\scshape\centering}
\renewcommand{\afterpartskip}{\relax}
\chapterstyle{veelo}
    \renewcommand*{\printchapternum}{%
    \makebox[0pt][l]{%
    \hspace{.8em}%
    \resizebox{!}{\beforechapskip}%
    {\chapnumfont \thechapter}%
    \hspace{.8em}%
    \rule{2\midchapskip}{\beforechapskip}%
    }%
}

% Hyperlinks
\usepackage[colorlinks, linkcolor = blue]{hyperref}
\usepackage{cleveref}

% Definition/Theorems, and Examples
\usepackage{tcolorbox}
\tcbuselibrary{theorems, breakable}
\newtcbtheorem[auto counter, crefname = {theorem}{theorems}, Crefname = {Theroem}{Theorems}]{thm}{Theorem}{
    sharp corners, colback = lightgray!40, breakable
}{thm}
\newtcbtheorem[auto counter, crefname = {axiom}{axioms}, Crefname = {Axiom}{Axioms}]{axiom}{Axiom}{
    sharp corners, colback = lightgray!40, breakable
}{axiom}
\newtcbtheorem[auto counter, crefname = {definition}{definition}, Crefname = {Definition}{Definition}]{df}{Definition}{
    sharp corners, colback = lightgray!40, colframe = darkgray, breakable
}{df}
\newtcbtheorem[auto counter, crefname = {lemma}{lemma}, Crefname = {Lemma}{Lemma}]{lemma}{Lemma}{
    sharp corners,
}{lemma}
\newtcbtheorem[auto counter, number within = section]{exmp}{Example}{
    colback = lightgray!40, colframe = darkgray, breakable
}{exmp}
\newtcbtheorem[auto counter, number within = chapter, crefname = {remarks of chapter }{remarks of chapter }, Crefname = {Remarks}{Remarks}]{remark}{Remarks on chapter }{
    colback = lightgray!10, colframe = black, breakable
}{remark}
% Proofs
\usepackage{amsthm}

% Conclusions
\newcommand{\conclusion}{\section{Conclusion for Chapter \thechapter}}
\newcommand{\formula}{\section{Formula from Chapter \thechapter}}
\newcommand{\prelude}[1]{
    \chapter*{Prelude: #1}
    \addcontentsline{toc}{chapter}{Prelude: #1}
}
\newcommand{\chapterexercises}{\section{Exercises for Chapter \thechapter}}

%%% Notation Commands

\usepackage[]{siunitx}
\usepackage{physics}
\AtBeginDocument{\RenewCommandCopy\qty\SI}

% Geometry
\let\line\overline
% Mathematical constants
\newcommand{\e}{\symrm{e}}
\newcommand{\im}{\symrm{i}}
\newcommand{\cpi}{\symrm{\pi}}
\DeclareMathOperator*{\ssum}{\symrm{\Sigma}}
\DeclareMathOperator*{\Proj}{\symrm{Proj}}
\DeclareMathOperator*{\fgamma}{\symrm{\Gamma}}
% Vector notations
\newcommand{\vv}[1]{\pmb{\symrm{#1}}}
\newcommand{\conj}{^{\ast}}
\newcommand{\dagr}{^{\dag}}
\newcommand{\trnsp}{^{\intercal}}
\newcommand{\iden}{\symbb{I}}
\renewcommand\vdot\cdot
% e Unit vectors
\newcommand{\uv}[1]{\hat{\vv{e}}_{#1}}
% Kronecker delta and Dirac's delta function
\newcommand{\kdel}[1]{\symrm{\delta}_{#1}}
\newcommand{\ddel}{\symrm{\delta}}
% Discrete differences
\newcommand{\Dd}[1]{\symrm{\Delta}{#1}}
% Limit arrows
\newcommand{\appr}{\rightarrow}
% Physics quantities symbols
\newcommand{\lagr}{\mathcal{L}}
\newcommand{\haml}{\mathcal{H}}
\newcommand{\hilb}{\mathcal{E}}
% Path integral measures
\newcommand{\pintm}[1]{\mathcal{D}[#1]}

% Unit vectors i j k
\newcommand{\ihat}{\hat{\i}}
\newcommand{\jhat}{\hat{\j}}
\newcommand{\khat}{\hat{k}}
% Tensor products
\newcommand{\tensor}{\otimes}
% And, and or, in math expressions
\newcommand{\mathand}{\quad\textrm{and,}\quad}
\newcommand{\mathor}{\quad\textrm{or,}\quad}

% Set theory notations
\NewDocumentCommand{\comp}{}{^\complement}

\newcommand{\prerequisites}[1]{\textbf{Prerequisites:}~\emph{#1}}

% Bibliographies
\usepackage[
    backend = biber,
    style = phys,
    sorting = anyvt
]{biblatex}
\addbibresource{bibliography.bib}

% TikZ
\usepackage{tikz, pgfplots, tikz-3dplot}
\pgfplotsset{compat = 1.18}

\usetikzlibrary{external}
\tikzexternalize[prefix = tikz/]

\usetikzlibrary{calc}
\usetikzlibrary{angles, quotes}
\usetikzlibrary{patterns}
\usetikzlibrary{decorations.pathreplacing, calligraphy}
\usetikzlibrary{3d}

\tikzstyle{brace} = [decorate, decoration = {brace, amplitude = 5pt, raise = 2pt}]
\tikzstyle{mirrored brace} = [decorate, decoration = {brace, amplitude = 3pt, raise = 2pt, mirror}]
\tikzstyle{unit} = [ultra thick, blue, -stealth]
\tikzstyle{axis} = [thick, -stealth]
\tikzstyle{vector} = [-latex]
\tikzstyle{line} = [latex-latex]

\begin{document}

\title{The study of the dynamics of quantum bipartite entangled systems using Feynman path integrals}
\author{Puripat Thumbanthu \\ Advisor: Assoc. Prof. Ekapong Hirunsirisawat, Dr. Tanapat Deesuwan}
\date{May 16, 2024}

\maketitle

\tableofcontents*

\chapter{Generalization of the Gaussian integrals}

\section{Preliminary form}

A Gaussian integral is the integral of the Gaussian function $\exp[-x^2]$:
\begin{equation}
    \int_{\infty}^{-\infty} \exp[-x^2]\dd{x} = \sqrt{\cpi}.
\end{equation}
However, this is not that useful for path integrals, because it demands a more generalized form of this Gaussian function. Therefore, we focus on integrals of the form
\begin{equation}
    I_1 = \int_{\infty}^{-\infty} x^n\exp[-(ax^2 + bx + c)]\dd{x},\label{eq:generalized_gaussian}
\end{equation}
where $a$, $b$, and $c$ are constants and $n$ is a positive integer. This integral will be very useful later on when evaluating the Born expansion of the propagator.

Because we're just going to be dealing with integrals from negative to positive infinity, I shall omit the bounds of the integrals. Thus, every integral written with no bounds from now on are assumed to be a definite integral from negative to positive infinity.

\section{Evaluation}

To evaluate $I_1$ (\cref{eq:generalized_gaussian}), we shall complete the square first, then substitute the exponents to fit the form of the standard Gaussian integral.
\begin{align*}
    \int x^n\e^{-(ax^2 + bx + c)}\dd{x} &= \int x^n\exp[-a\left(x + \frac{b^2}{2a}\right)^2 + \frac{b^2}{4a} - c]\dd{x} \\
    &= \e^{\frac{b^2}{4a} - c}\int x^n\exp[-a\left(x + \frac{b^2}{2a}\right)]\dd{x}.
\end{align*}
Let
\begin{align}
    -u^2 &= -a\left(x + \frac{b^2}{2a}\right)^2 \\
    x &= \frac{u}{\sqrt{a}} - \frac{b^2}{2a} \\
    \dv{x}{u} &= \dv{u}(\frac{u}{\sqrt{a}} - \frac{b^2}{2a}) \\
    \dd{x} &= \frac{1}{\sqrt{a}}\dd{u}.
\end{align}
Then,
\begin{align}
    I_1 &= \e^{\frac{b^2}{4a} - c}\frac{1}{\sqrt{a}}\int\left(\frac{u}{\sqrt{a}} - \frac{b^2}{2a}\right)^n\e^{-u^2}\dd{u}. \\
    &= \e^{\frac{b^2}{4a} - c}\frac{1}{\sqrt{a}}\int\sum_{k = 0}^{n}{n \choose k}\left(\frac{b^2}{2a}\right)^{n - k}\left(\frac{u}{\sqrt{a}}\right)^k\e^{-u^2}\dd{u}. \\
    &= \e^{\frac{b^2}{4a} - c}\frac{1}{\sqrt{a}}\sum_{k = 0}^{n}\left[{n \choose k}\left(\frac{b^2}{2a}\right)^{n - k}\left(\frac{1}{\sqrt{a}}\right)^{k} \int u^k\e^{u^2}\dd{u}\right], \label{eq:generalized_gaussian-1}
\end{align}
i.e., $I_1$ can be written as a sum of
\begin{equation}
    I_0 = \int u^k\e^{-u^2}\dd{u}.
\end{equation}
Notice that when $n$ is odd, the integrand of $I_0$ is an odd function. Therefore, $I_0$ is zero whenever $n$ is odd. When $n$ is even however, the integrand is even. Thus, it can be simplified to
\begin{equation}
    2\int_{0}^{\infty} u^{2m}\e^{-u^2}\dd{u}
\end{equation}
where $k = 2m$. We then do another substitution by letting $-t = -u^2$. Thus, $u = \sqrt{t}$ and $\dd{u} = \flatfrac{\dd{t}}{2\sqrt{t}}$. Both infinity and zero aren't affected by a square root, therefore the bound doesn't change. Our integral then becomes
\begin{align}
    &2 \times \int_{0}^{\infty}t^{m}\e^{-t}\frac{1}{2\sqrt{t}}\dd{t}. \\
    &= \int_{0}^{\infty}t^{m - \frac{1}{2}}\e^{-t}\dd{t} \\
    &= \int_{0}^{\infty}t^{m + \frac{1}{2} - 1}\e^{-t}\dd{t} \\
    &= \fgamma\left(m + \frac{1}{2}\right) \\
    &= \fgamma\left(\frac{k + 1}{2}\right).
\end{align}
Since the integral evaluates to the gamma function for only even numbers, we add a term that makes $I_0 = 0$ when $n$ is odd:
\begin{equation}
    I_0 = \frac{1}{2}((-1)^k + 1)\fgamma\left(\frac{k + 1}{2}\right).
\end{equation}
Thus, from \cref{eq:generalized_gaussian-1},
\begin{equation}
    I_1 = \e^{\frac{b^2}{4a} - c}\frac{1}{\sqrt{a}}\sum_{k = 0}^{n}\left[{n \choose k}\left(\frac{b^2}{2a}\right)^{n - k}\left(\frac{1}{\sqrt{a}}\right)^{k} \frac{1}{2}((-1)^k + 1)\fgamma\left(\frac{k + 1}{2}\right)\right]
\end{equation}
To continue this, we then expand the combinatorics and use the formulas of arguments with half-integer real part to get
\begin{align}
    &\begin{multlined}
    \e^{\frac{b^2}{4a} - c}\frac{1}{\sqrt{a}}\sum_{k = 0}^{n}\frac{n!}{k!(n - k)!}\left(\frac{b^2}{2a}\right)^n\left(\frac{2a}{b^2}\right)^k\left(\frac{1}{\sqrt{a}}\right)^k\left(\sqrt{\cpi}\frac{k!}{4^{\frac{k}{2}}\left(\flatfrac{k}{2}\right)!}\right) \\
    \times \frac{1}{2}((-1)^k + 1)
    \end{multlined} \\
    &= \e^{\frac{b^2}{4a} - c}n!\sqrt{\frac{\cpi}{a}}\left(\frac{b^2}{2a}\right)^n\sum_{k = 0}^n\frac{1}{(n - k)!(\flatfrac{k}{2})!}\left(\frac{\sqrt{a}}{b^2}\right)^{k}\frac{1}{2}((-1)^k + 1)
\end{align}
To simplify the summation, we replace $k$ with $2k$, and the upper limit with $m$ where $m = \left\lfloor\flatfrac{n}{2}\right\rfloor$. Thus, we get
\begin{equation}
    \left[\sum_{k = 0}^{m}\frac{1}{(n - 2k)!k!}\left(\frac{\sqrt{a}}{b^2}\right)^{2k}\right]\e^{\frac{b^2}{4a} - c}n!\left(\frac{b^2}{2a}\right)^n\sqrt{\frac{\cpi}{a}}
\end{equation}
As far as I know, this equation cannot be simplified further.

\paragraph{Comment on the generalized Gaussian integral} Note that the integral in \cref{eq:generalized_gaussian} can be casted in the form of Meijer's $G$ function for $n \geq 0$:
\begin{multline}
    \int x^n\exp[-ax^2 - bx - c]\dd{x} \\= \frac{e^{-c}}{2\sqrt{\cpi}ab}\left[- \frac{2^{n + 1}a}{b^n} {G_{2, 1}^{1, 2}\left(\begin{matrix} \frac{1 - n}{2}, - \frac{n}{2}  \\0 \end{matrix} \middle| {\frac{4 a e^{- 2 i \pi}}{b^{2}}} \right)} + a^{\frac{1 - n}{2}}b {G_{1, 2}^{2, 1}\left(\begin{matrix} \frac{1 - n}{2}\\0, \frac{1}{2}\end{matrix} \middle| {\frac{b^{2}}{4 a}} \right)}\right]
\end{multline}
However, this form doesn't seem to be very helpful when we try to evaluate the polynomial part of the Gaussian integral because of the nature of the Meijer's $G$ function that's defined with integral of products.

\section{Tables of Gaussian integrals with varying orders}
\label{sec:table_of_gaussian_integrals}

\everymath{\displaystyle}
\begin{enumerate}[label = {$n = \arabic*$:}]
    \setcounter{enumi}{-1}
    \item $+\frac{1}{1}\sqrt{\frac{\cpi}{a}}\e^{\frac{b^2}{4a} - c}$
    \item $-\frac{b}{2}\sqrt{\frac{\cpi}{a^3}} \e^{\frac{b^2}{4a} - c}$
    \item $+\frac{1}{4}\sqrt{\frac{\cpi}{a^5}} (2a + b^2) \e^{\frac{b^2}{4a} - c}$
    \item $-\frac{b}{8}\sqrt{\frac{\cpi}{a^7}} (6a + b^2) \e^{\frac{b^2}{4a} - c}$
    \item $+\frac{1}{16}\sqrt{\frac{\cpi}{a^9}} (12a^2 + 12ab^2 + b^4)\e^{\frac{b^2}{4a} - c}$
    \item $-\frac{b}{32}\sqrt{\frac{\cpi}{a^{11}}} (60a^2 + 20ab^2 + b^2)\e^{\frac{b^2}{4a} - c}$
    \item $+\frac{1}{64}\sqrt{\frac{\cpi}{a^{13}}} (120a^3 + 180a^2b^2 + 30ab^4 + b^6)\e^{\frac{b^2}{4a} - c}$
    \item $-\frac{b}{128}\sqrt{\frac{\cpi}{a^{15}}} (840a^3 + 420a^2b^2 + 42ab^4 + b^6)\e^{\frac{b^2}{4a} - c}$
    \item $+\frac{1}{256}\sqrt{\frac{\cpi}{a^{17}}} (1680a^4 + 3360a^3b^2 + 840a^2b^4 + 56ab^6 + b^8)\e^{\frac{b^2}{4a} - c}$
    \item $-\frac{b}{512}\sqrt{\frac{\cpi}{a^{19}}} (15120a^4 + 10080a^3b^2 + 1512a^2b^4 + 72ab^6 + b^8)\e^{\frac{b^2}{4a} - c}$
\end{enumerate}
\chapter{Propagator and joint probability distribution}

\section{General theory for the propagator of two particles}

\subsection{Non-interacting systems}

The propagator for a single particle system is used to find the probability distribution $\tilde{P}(x)$ of a state at anytime, i.e., for a state $\ket{\psi(t_0)}$,
\begin{equation}
    \tilde{P}(q_F, t_F) = \braket{q_F}{\psi(t_F)} = \int(q_0, \Dd{t})\psi(q_0, t_0)\dd{q_0}K.
\end{equation}
When taken the modulus squared, $|\tilde{P}(q_F, t_F)|^2$, or just $P(q_F, t_F)$, represents the probability of finding that particle at the point $(q_F, t_F)$ in spacetime.

We shall then extend this idea to describe a two particle system. Let there be a state ket $\ket{\eta(t)}$, which simultaneously represents the state of both particle one and particle two. Assume that at some point in time, $\ket{\eta}$ is separable:
\begin{equation}
    \ket{\eta(t_0)} = \ket{\psi(t_0)} \tensor \ket{\phi(t_0)}.
\end{equation}
If the subsystem of $\ket{\eta}$ is non-interacting, the Hamiltonian must be completely separable, which also implies that the time evolution operator $\hat{U}$ is also separable. Consider $\braket{q_F, q_F'}{\eta(t_F)}$ where the unprimed $q$ belongs to the position in the Hilbert space of particle one, and the primed $q$'s, in Hilbert space of particle two.
\begin{align*}
    \braket{q_F, q_F'}{\eta(t_F)} &= \bra{q_F, q_F'}\left(\ket{\psi_i(t_F)}\ket{\phi_i(t_F)}\right) \\
    &= \psi_i(q_F, t_F)\phi_i(q_F', t_F') \\
    &= \iint K_{\psi}(q_F, t_F; q_0, t_0)K_{\phi}(q_F', t_F'; q_0', t_0')\left[\psi_i(q_F, t_F)\phi_i(q_F', t_F')\right] \dd{q_0}\dd{q_0'}.
\end{align*}
For most systems, $K_{\psi}$ and $K_{\phi}$ is identical because both $\psi_i$ and $\phi_i$ is affected by the same potential. But for some, $K_{\psi}$ and $K_{\phi}$ might not be identical. E.g., if $\psi$ is in a potential well, but $\phi$ is a free particle that's infinitely far away. $K_{\psi}$ must be of the potential well, and $K_{\phi}$ must be of the free particle.

It can be seen that $\braket{q_F, q_F'}{\eta(t_F)}$ represents the probability distribution of both the two particles. E.g., for a state ket $\ket{\psi}\tensor\ket{\phi}$,
\begin{align}
    & (\bra*{q_F}\bra*{q_F'})(\ket*{\psi(t_F)}\ket*{\phi(t_F')}) \\
    &= \braket*{q_F}{\psi(t_F)}\braket*{q_F'}{\phi(t_F')} \label{eq:prejointprobability1}\\
    &= \iint\dd{q_0}\dd{q_0'}K(q_F, t_F; q_0, t_0)K(q_F', t_F';q_0', t_0')(\psi(q_0, t_0)\phi(q_0', t_0')) \\
    &= \int\dd{q_0}K(q_F, t_F; q_0, t_0)\psi(q_0, t_0)\int\dd{q_0'}K(q_F', t_F';q_0', t_0')\phi(q_0', t_0'), \label{eq:prejointprobability2}
\end{align}
\cref{eq:prejointprobability1} says that $\braket{q_F, q_F'}{\eta(t_F)}$ is the product of the probability that $\psi$ is at $(q_F, t_F)$, and $(q_F', t_F')$. Therefore, $\braket{q_F, q_F'}{\eta(t_F)}$ represents the joint probability distribution of the two states, and it can be found via the product of propagators,
\begin{equation}
    K(q_F, t_F; q_0, t_0)K(q_F', t_F';q_0', t_0').
\end{equation}


\subsection{Interacting systems}

Let there be a state ket $\ket{\eta}$ which describes the quantum state of two particles. When the two particles are interacting, the time-evolution operator cannot be factored into tensor products of two operators. Therefore, there can only be one combined time-evolution operator for both of the systems:
\begin{equation}
    \hat{U}(t_F, t_0) = \exp[-\im\Dd{t}\left(\frac{\hat{p}^2}{2m} + \frac{\hat{p}'^2}{2m'} + V(\hat{q}, \hat{q}', t)\right)]
\end{equation}
The consequence of the combined time-evolution operator is that, we cannot calculate the joint probability distribution between different time of the subsystem. For simplicity’ sake, let the index $j = i + 1$, where $t_j = t_i + \Dd{t}$ in which $\Dd{t} \appr 0$.
\begin{align}
    &\braket{q_j, q_j'}{\eta(t_j)} \\
    &= \mel{q_j, q_j'}{\hat{U}(t_j, t_i)}{\eta(t_i)}. \\
    &= \iint \dd{q_i}\dd{q_i'} \mel{q_j, q_j'}{\hat{U}(t_j, t_i)}{q_i, q_i'}\braket{q_i, q_i'}{\eta(t_i)}.
\end{align}

As seen, the form of transition element is
\begin{equation}
    \mel{q_j, q_j'}{\exp[-\im\Dd{t}\left(\frac{\hat{p}^2}{2m} + \frac{\hat{p}'^2}{2m'} + V(\hat{q}, \hat{q}', t)\right)]}{q_i, q_i'}.
\end{equation}
The terms in the exponents can be separated due to the vanishing commutator when $\Dd{t} \appr 0$. By separating the terms in the exponents and inserting two complete sets of momentum basis, the equation above turns into
\begin{align}
    & \frac{1}{(2\cpi)^2} \iint\dd{p}\dd{p'} \mel{q_j, q_j'}{\exp[-\im\Dd{t}\frac{\hat{p}^2}{2m}]}{p, p'}\mel{p, p'}{\exp[-\im\Dd{t}\frac{\hat{p}'^2}{2m'}]\exp[-\im\Dd{t}V(\hat{q}, \hat{q}', t)]}{q_i, q_i'} \nonumber\\
    &= \frac{1}{(2\cpi)^2} \iint\dd{p}\dd{p'} \exp[-iu\Dd{t}\left(\frac{p^2}{2m} + \frac{p'^2}{2m'} + V(q_i, q_i, \Dd{t})\right)]\braket{q_j}{p}\braket{q_j'}{p'}\braket{q_i}{p}\conj\braket{q_j'}{p'}\conj \nonumber\\
    &= \frac{1}{(2\cpi)^4} \e^{-\im\Dd{t}V(q_i, q_i', t)} \int\dd{p}\exp[-\im\Dd{t}\frac{p^2}{2m} + \im p(q_j - q_i)]\int\dd{p'}\exp[-\im\Dd{t}\frac{p'^2}{2m'} + \im p'(q_j' - q_i')^2] \nonumber\\
    &= \frac{(mm')^{\frac{1}{2}}}{8\cpi^3\im\Dd{t}}\exp[-\im\Dd{t}V(q_i, q_i', t) + \frac{\im m'}{2\Dd{t}}(q_j' - q_i')^2 + \frac{\im m}{2\Dd{t}}(q_j - q_i)^2].
\end{align}
To find the propagator for an interacting system, we need to perform successive integrals on $q$ and $q'$, i.e.,
\begin{equation}
    K_{\eta} = \idotsint \dd{q_N}\dd{q_N'}\dots\dd{q_1}\dd{q_1'}\mel{q_F, q_F'}{\hat{U}(t_F, t_N)}{q_N, q_N'}\dots\mel{q_1, q_1'}{\hat{U}(t_1, t_0)}{q_0, q_0'}
\end{equation}

Notice that when there is no interaction between the two systems ($V = 0$), the integrals become separable and reduces down to the form of the non-interacting system's propagator but off by a normalization factor.

There are two common forms of interaction, which is the spring interaction and the coulomb interaction. Both of which includes the term $(q_i - q_i')^2$ in $V(q, q')$, which causes major problems in integration. When expanded, there is a $q_iq_i'$ term that makes the integral inseparable which causes the integral pattern to not repeat; therefore, we resort to perturbation.

\section{Form of the two particle perturbation series}

The perturbation series for one particle are already given by Feynman in his path integrals textbook:
\begin{equation}
    K_n(F, 0) = \left(-\im\right)^n\idotsint K_0(F, n)\prod_{i = 1}^n V(i)K(i, i - 1)\dd{\tau_i}
\end{equation}
where $K(j, k) = K(q_j, t_j; q_k, t_k)$. To apply it with two particles, we extend it:
\begin{equation}
    K_n(F, 0; F', 0') = \left(-\im\right)^n\idotsint K_0(F, n)K_0(F', n')\left[\prod_{i = 1}^nV(i)K(i, i - 1)K(i', i - 1')\dd{q_i}\dd{q_i'}\right]\dd{t}. \label{eq:generaltwoparticleperturbation}
\end{equation}
where $K = \ssum K_n$. For some potential, it is possible to evaluate this series term by term analytically using the Meijer's $G$ function.

\section{The spring system}

The spring potential is given by $V(q, q') = \flatfrac{k}{2}(q - q')^2$, which I shall let $\alpha = \flatfrac{k}{2}$ for simplicity sake.

\subsection{First order perturbation term}
\label{sec:spring_1storder}

From \cref{eq:generaltwoparticleperturbation}, set $t_1 = t, t_0 = 0$
\begin{align}
    &K_1(F, 0; F', 0) \nonumber \\
    &= \left(-\im\right)^1\iiint K_0(F, 1)K_0(F', 1')K_0(1, 0)K_0(1, 0')\alpha(q_1 - q_1')^2\dd{q_1}\dd{q_1'}\dd{t}. \\
    &= -\im\alpha\int_0^{t_F}\left[\iint K_0(q_F, q_1; t_F - t)K_0(q_F', q_1'; t_F - t)K_0(q_1, q_0; t)K_0(q_1', q_0'; t)(q_1 - q_1')^2\dd{q_1}\dd{q_1'}\right]\dd{t}. \nonumber
\end{align}
We then let the terms in the square bracket,
\begin{equation}
    \iint K_0(q_F, q_1; t_F - t)K_0(q_F', q_1'; t_F - t)K_0(q_1, q_0; t)K_0(q_1', q_0'; t)(q_1 - q_1')^2\dd{q_1}\dd{q_1'}
\end{equation}
equals $I$; therefore, $K_1 = -\im\alpha\int_0^{t_F} I\dd{t}$.

We then separate $I$ into three integrals:
\begin{gather}
    I_{P1} = \int q_1^2K_0(q_F, q_1; t_F - t)K_0(q_1, q_0; t)\dd{q_1} \int K_0(q_F', q_1'; t_F, t)K_0(q_1', q_0'; t)\dd{q_1'}, \label{eq:spring_IP1}\\
    I_{P2} = \int K_0(q_F, q_1; t_F - t)K_0(q_1, q_0; t) \int q_1'^2K_0(q_F', q_1'; t_F, t)K_0(q_1', q_0'; t)\dd{q_1'}, \label{eq:spring_IP2}\\
    I_{P3} = \int q_1K_0(q_F, q_1; t_F - t)K_0(q_1, q_0; t) \int q_1'K_0(q_F', q_1'; t_F, t)K_0(q_1', q_0'; t) \dd{q_1'} \label{eq:spring_IP3}
\end{gather}
where $I = I_{P1} + I_{P2} + 2I_{P3}$. The integrals without the factor $q_1$ and $q_1^2$ can be reduced into the kernel for the free particle:
\begin{gather}
    I_{P1} = K_0(q_F', q_0'; t_F)\int q_1^2K_0(q_F, q_1; t_F - t)K_0(q_1, q_0; t)\dd{q_1} \\
    I_{P2} = K_0(q_F, q_0; t_F)\int q_1'^2 K_0(q_F', q_1'; t_F - t)K_0(q_1', q_0'; t)\dd{q_1'}
\end{gather}
Since $I_{P2}$ can be obtained by switching all the primed variables with the corresponding unprimed in $I_{P1}$, we're left with two family of integrals:
\begin{equation}
    \int q_1K_0(q_F, q_1; t_F - t)K_0(q_1, q_0; t)\dd{q_1} \mathand \int q_1K_0(q_F, q_1; t_F - t)K_0(q_1, q_0; t)\dd{q_1}. \label{eq:spring_family_of_integrals}
\end{equation}
To evaluate these, we first simplify the product of kernel under the assumption that $t_F > t$.
\begin{align}
    &K_0(q_F, q_1; t_F - t)K_0(q_1, q_0; t) \nonumber\\
    &= \sqrt{\frac{m}{2\cpi\im(t_F - t)}}\sqrt{\frac{m}{2\cpi\im t}}\exp[-\frac{\im m}{2(t_F - t)}(q_F - q_1)^2 - \frac{\im m}{2t}(q_1 - q_0)^2] \\
    &= \frac{m}{2\cpi}\sqrt{-\frac{1}{t(t_F - t)}} \exp[q_1^2\left(\frac{\im m}{2t} + \frac{\im m}{2(t_F - t)}\right) - q_1\left(\frac{\im m q_0}{t} + \frac{\im m q_F}{t_F - t}\right) + \left(\frac{\im m q_0^2}{2t} + \frac{\im m q_F^2}{2(t_F - t)}\right)] \nonumber \\
    &= \frac{m}{2\cpi}\sqrt{-\frac{1}{t(t_F - t)}} \exp[-q_1^2\left(\frac{m t_F}{2\im t(t_F - t)}\right) - q_1(\im m)\left(\frac{q_0}{t} + \frac{q_F}{t_F - t}\right) - \left(\frac{m}{2\im}\right)\left(\frac{q_0^2}{t} + \frac{q_F^2}{t_F - t}\right)] \nonumber
\end{align}
The normalization factor are pulled out. Both integrals in \cref{eq:spring_family_of_integrals} can be evaluated with
\begin{equation}
    a = \frac{m t_F}{2\im t(t_F - t)}, \quad b = \im m\left(\frac{q_0}{t} + \frac{q_F}{t_F - t}\right) \mathand c = \left(\frac{m}{2\im}\right)\left(\frac{q_0^2}{t} + \frac{q_F^2}{t_F - t}\right)
\end{equation}
in which,
\begin{equation}
    \exp[\frac{b^2}{4a} - c] = \exp[\frac{\im m}{2t_F}(q_F - q_0)^2] = \sqrt{\frac{2\cpi\im t_F}{m}}K_0(q_F, q_0; t_F).
\end{equation}
To summarize,
\begin{align}
    \int q_1K_0(q_F, q_1; t_F - t)K_0(q_1, q_0; t)\dd{q_1} &= -\frac{b}{2}\sqrt{\frac{\cpi}{a^3}}\times\frac{m}{2\cpi}\sqrt{-\frac{1}{t(t_F - t)}}\times\sqrt{\frac{2\cpi\im t_F}{m}}K_0(q_F, q_0; t_F) \nonumber\\
    &= -\frac{b}{2}\sqrt{\frac{\cpi}{a^3}} \times \sqrt{\frac{mt_F}{2\cpi\im t(t_F - t)}}K_0(q_F, q_0; t_F) \label{eq:spring_deg1gaussian_form}
\end{align}
and,
\begin{equation}
    \int q_1K_0(q_F, q_1; t_F - t)K_0(q_1, q_0; t)\dd{q_1} = \frac{1}{4}\sqrt{\frac{\cpi}{a^5}}(2a + b^2) \times \sqrt{\frac{mt_F}{2\cpi\im t(t_F - t)}}K_0(q_F, q_0; t_F). \label{eq:spring_deg2gaussian_form}
\end{equation}

On the Gaussian integral with degree one,
\begin{equation}
    -\frac{b}{2}\sqrt{\frac{\cpi}{a^3}} = \sqrt{\frac{2\cpi t}{mt_F^3}}\frac{\sqrt{-\im(t_F - t)^3}}{\im(t_F - t)} \times \left[q_0(t_F - t) + q_Ft\right]
\end{equation}
;thus from \cref{eq:spring_deg1gaussian_form},
\begin{align}
    \int q_1K_0(q_F, q_1; t_F - t)K_0(q_1, q_0; t) &= \begin{multlined}[t]
        \sqrt{\frac{2\cpi t}{mt_F^3}}\frac{\sqrt{-\im(t_F - t)^3}}{\im(t_F - t)} \times \left[q_0(t_F - t) + q_Ft\right] \\
        \times \sqrt{\frac{mt_F}{2\cpi\im t(t_F - t)}}K_0(q_F, q_0; t_F)
    \end{multlined} \\
    &= -\frac{1}{t_F}\left[q_0(t_F - t) + q_Ft\right]K_0(q_F, q_0; t_F).
\end{align}
On the Gaussian integral with degree two,
\begin{equation}
    \frac{1}{4}\sqrt{\frac{\cpi}{a^5}}(2a + b^2) = -\sqrt{\frac{2\cpi t}{m^3t_F^5}}\frac{\sqrt{\im(t_F - t)^5}}{(t_F - t)^2}\left[m\left(q_0(t_F - t) + q_Ft\right)^2 + \im tt_F(t_F - t)\right];
\end{equation}
and thus,
\begin{align}
    &\int q_1K_0(q_F, q_1; t_F - t)K_0(q_1, q_0; t)\dd{q_1} \\
    &= \sqrt{\frac{mt_F}{2\cpi\im t(t_F - t)}}K_0(q_F, q_0; t_F)  \times -\sqrt{\frac{2\cpi t}{m^3t_F^5}}\frac{\sqrt{\im(t_F - t)^5}}{(t_F - t)^2}\left[m\left(q_0(t_F - t) + q_Ft\right)^2 + \im tt_F(t_F - t)\right] \nonumber\\
    &= -\frac{1}{mt_F^2}\left[m\left(q_0(t_F - t) + q_Ft\right)^2 + \im tt_F(t_F - t)\right]K_0(q_F, q_0; t_F)
\end{align}

\subsection{Second order perturbation term}

The second order perturbation term, $K_2$ is
\begin{align}
    K_2 &= \idotsint K_0(F, 2)V(2)K_0(2, 1)V(1)K_0(1, 0)\dd{q_1}\dd{q_1'}\dd{t_1}\dd{q_2}\dd{q_2'}\dd{t_2} \\
    &= \begin{multlined}[t]
        \idotsint K_0(q_F, q_2; t_F - t_2)K_0(q_F', q_2'; t_F - t_2)K_0(q_2, q_1; t_2 - t_1)K_0(q_2, q_1; t_2 - t_1) \\ \times K_0(q_1, q_0; t_1 - t_0)K_0(q_1, q_0; t_1 - t_0)(q_2 - q_2')^2(q_1 - q_1')^2\dd{q_1}\dd{q_1'}\dd{t_1}\dd{q_2}\dd{q_2'}\dd{t_2}
    \end{multlined} \\
    &= \begin{multlined}[t]
        \int_{t_1}^{t_F}\int_{t_0}^{t_F}\left[\idotsint K_0(q_F, q_2; t_F - t_2)K_0(q_F', q_2'; t_F - t_2)K_0(q_2, q_1; t_2 - t_1)\right. \\ \times K_0(q_2, q_1; t_2 - t_1)K_0(q_1, q_0; t_1 - t_0)K_0(q_1, q_0; t_1 - t_0) \left(\vphantom{\left(q_{2}^{\prime}\right)^{2}} q_{1}^{2} q_{2}^{2} - 2 q_{1}^{2} q_{2} q_{2}^{\prime} \right.\\ + q_{1}^{2} \left(q_{2}^{\prime}\right)^{2} - 2 q_{1} q_{2}^{2} q_{1}^{\prime} + 4 q_{1} q_{2} q_{1}^{\prime} q_{2}^{\prime} - 2 q_{1} q_{1}^{\prime} \left(q_{2}^{\prime}\right)^{2} + q_{2}^{2} \left(q_{1}^{\prime}\right)^{2} \\ \left.\left. - 2 q_{2} \left(q_{1}^{\prime}\right)^{2} q_{2}^{\prime} + \left(q_{1}^{\prime}\right)^{2} \left(q_{2}^{\prime}\right)^{2} \right) \dd{q_1}\dd{q_1'}\dd{q_2}\dd{q_2'} \vphantom{\idotsint}\right]\dd{t_1}\dd{t_2}
    \end{multlined}
\end{align}
The integral once again can be broken into nine integrals that must be integrated w.r.t. time twice later on. All those nine integrals have a product of propagator as a multiplier. We shall evaluate those first, separating the primed and unprimed variables.
\begin{align}
    &\begin{multlined}[t]
        K_0(q_F, q_2; t_F - t_2)K_0(q_F', q_2'; t_F - t_2)K_0(q_2, q_1; t_2 - t_1)K_0(q_2, q_1; t_2 - t_1)K_0(q_1, q_0; t_1 - t_0) \\ \times K_0(q_1, q_0; t_1 - t_0)
    \end{multlined} \nonumber \\
    &= \begin{multlined}[t]
        \frac{\im m^3}{8\cpi^3(t_F - t_2)(t_2 - t_1)(t_1 - t_0)}\exp \left[\frac{\im m}{2(t_F - t_2)}\left((q_F - q_2)^2 + (q_F' - q_2')^2\right) \right. \\ \left. + \frac{\im m}{2(t_2 - t_1)}\left((q_2 - q_1)^2 + (q_2' - q_1')^2\right) + \frac{\im m}{2(t_1 - t_0)}\left((q_1 - q_0)^2 + (q_1' - q_0')^2\right)\right]
    \end{multlined}
\end{align}

\subsection{The joint probability distribution}

\section{The delta function collision problem}

The potential for the delta function collision problem is
\begin{equation}
    V = V_0\ddel(q - q')
\end{equation}
where $V_0$ is the strength of the delta function, and is generally considered to be negative.

\subsection{First order perturbation term}

We evaluate the perturbation term similarly to how we did it in \cref{sec:spring_1storder}, starting with the form:
\begin{align}
    &K_1(F, 0; F', 0) \\
    &= -\im\int_0^{t_F}\left[\iint K_0(q_F, q_1; t_F - t)K_0(q_F', q_1'; t_F - t)K_0(q_1, q_0; t)K_0(q_1', q_0'; t)V_0\ddel(q_1 - q_1')\dd{q_1}\dd{q_1'}\right]\dd{t}. \nonumber \\
    &= -\im V_0\int_0^{t_F}\left[\iint K_0(q_F, q_1; t_F - t)K_0(q_F', q_1'; t_F - t)K_0(q_1, q_0; t)K_0(q_1', q_0'; t)\ddel(q_1 - q_1')\dd{q_1}\dd{q_1'}\right]\dd{t} \nonumber
\end{align}
Let $I_1$ represents the integral in the square bracket; thus, $K_1 = -\im V_0\int_{0}^{t_F}I_1\dd{t}$. Then,
\begin{align}
    I_1 &= \iint K_0(q_F, q_1; t_F - t)K_0(q_F', q_1'; t_F - t)K_0(q_1, q_0; t)K_0(q_1', q_0'; t)\ddel(q_1 - q_1')\dd{q_1}\dd{q_1'} \\
    &= \int K_0(q_F, q_1'; t_F - t)K_0(q_F', q_1'; t_F - t)K_0(q_1', q_0; t)K_0(q_1', q_0'; t)\dd{q_1'}
\end{align}

\section{The coulomb problem}
\chapter{The Schr\"odinger's equation for two particles}

In this chapter, we'd try to formulate the same interaction problems in terms of the Schr\"odinger's equation in the non-moving frame of reference. Consider the Hamiltonian
\begin{equation}
    \haml(q, q') = -\frac{1}{2m}\pdv[2]{q} -\frac{1}{2m'}\pdv[2]{(q')} + V(q, q')
\end{equation}
The Schr\"odinger's equation then becomes
\begin{equation}
    \im\pdv{t}\eta(q, q'; t) = -\frac{1}{2m}\pdv[2]{\eta(q, q'; t)}{q} -\frac{1}{2m'}\pdv[2]{\eta(q, q'; t)}{(q')} + V(q, q')
\end{equation}

\end{document}